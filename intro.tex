

\subsection{Background and motivation}
We begin with a concise overview of the study of representation and conjugacy class growth, situating the present work within the broader context of current research in these areas. This is followed by a brief summary of the development of zeta functions with emphasis on those we cover in this paper. We extend this to introduce Euler products decompositions, bivariate zeta functions, and Lie lattices, which together form the framework necessary for stating our main results.


\subsubsection{Representation growth}
A representation of a group $G$ is a pair $(V, \rho)$ consisting of a vector space $V$ and group a homomorphism $\rho : G\to \mathrm{GL}_n(k)$. When $G$ is finite, all representations are semisimple and Maschke's theorem allows us to decompose them into irreducible representations. There are only finitely many of these, determined by their character (the trace function of $\rho$). For certain infinite groups there exist generalisations of Maschke's theorem, for example the Peter-Weyl theorem (cf. \cite[Theorem 4.1]{Bump2013LieGroups}) for compact Lie groups. In general, when $G$ is not finite it is more difficult to classify representations, and key areas of interest include non-compact Lie groups (via highest weight theory), and profinite groups (given by Galois groups of infinite degree field extensions, cf.~\cite{Waterhouse1974}). In this paper, we will consider the representation growth of $\mathcal T$-groups, which are finitely generated, nilpotent, and torsion-free.

\subsubsection{Conjugacy class growth}\label{conjdiscussion}
Conjugacy classes are equivalence classes partitioned by the relation "equal under conjugation" by elements of the group. Again, when $G$ is finite these are easily found and classified. In fact, the number of conjugacy classes is precisely the number of irreducible representations over $\mathbb C$ (up to isomorphism). For studying infinite groups, a key theorem to note is Neumann's theorem, which states that there exists an upper bound to all conjugacy class sizes if and only if the derived subgroup (the subgroup formed by commutators) is finite \cite[Theorem 3.1]{Neumann1954}.

In this paper we look at the growth of a counting function for conjugacy classes of a given cardinality, however other growth functions exist and are also key areas of interest. For example, let $G$ be a finitely generated group generated by a set $S$. We can define \textit{word length}, where elements of the group are represented by words formed from the alphabet $S\cup S^{-1}$. The \textit{Cayley graph} encodes certain structural properties of finitely generated groups, and we say an element lies in the ball of radius $n$ (of the Cayley graph, centred at the identity) if it has a representative of length $\leq n$. Common growth functions include counting the number of conjugacy classes that intersect the ball of radius $n$, as well as the number of elements within a given class that fall inside this ball. These have been proven to have polynomial growth for certain groups, see \cite{conjugacyclassgrowthvirtually} and \cite{rationalgrowthvirtuallyabelian}.

A remark in \cite{GrunewaldSegalSmith1988} posed the question of studying zeta functions of the conjugacy class-counting function, specifically for $\mathcal T$-groups. Modifications of this function exist for when our groups have infinite conjugacy classes of a given size; we will explore these later.


\subsubsection{Zeta functions}
Zeta functions of groups were introduced by \\Grunewald, Segal, and Smith~\cite{GrunewaldSegalSmith1988} in order to study subgroup growth of finitely generated groups. In their paper, zeta functions are given by a Dirichlet series
$$\zeta_G(s)=\sum_{H\leq G}|G:H|^{-s}.$$
%
More specifically they studied $\mathcal T$-groups, counting their finite index normal subgroups and conjugacy classes of subgroups. This was continued by Hrushovski et al.~\cite{handm} who studied representation zeta functions of $\mathcal T$-groups, counting irreducible representations up to (twist equivalence) of finite-dimensional complex characters.

An \textit{Euler product} is a decomposition of a Dirichlet series into an infinite product, indexed by primes. This was originally proven for the series 
%
$$\sum^{\infty}_{n=0}\frac{1}{n^s}= \prod_p\frac{1}{1-p^{-s}},$$
%
which later became known as the \textit{Riemann zeta function}, $\zeta(s)$. Most zeta functions satisfy these Euler products, including all the ones mentioned in this paper. By decomposing them into an Euler product, we can study their local properties. We have 
%
$$\zeta_G(s) = \prod_{p}\zeta_{G,p}(s),$$
%
for primes $p$, where $\zeta_{G,p}(s)$ sums over finite $p$-power index subgroups. 

Given a number field $K$ with a ring of integers $\mathcal O$, an $\mathcal O$-Lie lattices is a free and finitely generated $\mathcal O$-module equipped with a Lie bracket (a bilinear, antisymmetric operation satisfying the Jacobi identity). It was shown by Grunewald et al.~\cite{GrunewaldSegalSmith1988} that, given a $\mathcal T$-group $G$, we have an associated $\mathbb Z$-Lie lattice $L$, such that for almost all (all but finitely many) primes $p$, we have
%
\begin{align}\label{LIE}\zeta_{G,p}(s) = \zeta_{L,p}(s).\end{align}
%
We currently only have sufficient but not necessary conditions on the primes $p$ for which this holds. In \cite{voll2008}, Voll uses this to prove that local functional equations hold for almost all factors of representation and conjugacy class zeta functions of $\mathcal T$-groups. The relation \eqref{LIE} is a key result used in much subsequent research into $\mathcal T$-groups.

Let $G$ be a group, and denote by $r_n(G)$ for $n\in\mathbb N$, the number of isomorphism classes of $n$-dimensional irreducible representations. When $r_n(G)$ is finite we call $G$ \textit{representation rigid}, and we can encode it into a representation zeta function in order to study the growth of $r_n(G)$. Similarly, we define $c_n(G)$ to be the number of conjugacy classes of $G$ with cardinality $n$, and when this is finite $G$ is \textit{conjugacy rigid}. These functions are defined as follows.

\begin{define}\label{zetaorigdef}
    The representation zeta function and the conjugacy class zeta function for a group $G$ are, respectively
    \begin{align*}
        &\zeta^\mathrm{irr}_G(s)= \sum^\infty_{n=1}r_n(G)n^{-s} &\text{and}&&
        \zeta^\mathrm{cc}_G(s)= \sum^\infty_{n=1}c_n(G)n^{-s}.
    \end{align*}
\end{define}

Recall our focus on $\mathcal T$-groups, which are finitely generated, torsion-free, and nilpotent. They are neither representation rigid nor conjugacy rigid, hence we need to alter the zeta functions in order to study them.

In \cite{handm}, Hrushovski et al. introduced \textit{twist equivalent} representations, classes of which $\mathcal T$-groups have finitely many. These differ by tensoring with a one-dimensional representation of $G$ ie. two representations $\rho$ and $\sigma$ are twist equivalent if $\rho \cong \chi \otimes\sigma$ where $\chi$ is a one dimensional (and therefore irreducible) representation of $G$. Under this equivalence, we call classes of representations \textit{twist iso-classes}. We denote by $\widetilde{r}_n(G)$ the number of twist iso-classes of dimension $n$ in $G$. The updated zeta function is
%
\begin{align}\label{twistzeta}  
\zeta_G^{\widetilde{\mathrm{irr}}}(s) = \sum^\infty_{n=1} \widetilde{r}_n(G)n^{-s}.\end{align}
%
The fact that $\widetilde{r}_n(G)$ grows polynomially follows from the PSG (polynomial subgroup growth) Theorem \cite[Theorem A]{LubotzkyMann1987}, and \cite[Lemma 2.1]{zeta1}, hence this sum converges on a right half of $\mathbb C$. It is interesting to note that twist equivalence is also studied for conjugacy growth, where two elements $x,y \in G$ are \textit{twist conjugate} if $y=g\cdot x \cdot \phi(g^{-1})$, with $g \in G $ and $\phi \in \mathrm{Aut}(G)$ (cf.~\cite{twistconj,dekimpe2024twistedconjugacygrowthvirtually}), however this is not covered in this paper.

% \textcolor{red}{here!}rational in \cite{rnrational}.
As before, let $K$ be a number field and $\mathcal O$ its ring of integers. Then we can obtain $\mathcal{T}$-groups from unipotent group schemes $\mathbf G$ over $\mathcal O$ (cf. Section \ref{sec2}). All the $\mathcal T$-groups considered in this paper -- and moreover, all $\mathcal T$-groups of nilpotency class 2, which we will call $\mathcal T_2$-groups -- can be constructed in this way. Throughout, we shall refer to $\mathcal T$-groups by their form $\mathbf G(\mathcal O)$.

An extension of the zeta functions \eqref{zetaorigdef} are the bivariate zeta functions introduced by Lins in \cite{LinsI}. To overcome the issue of rigidity, we look at $r_n(Q)$ and $c_n(Q)$ for principle congruence quotients $Q$ of $\mathbf G (Q)$, and define the following. 
%
\begin{define}[{\cite[Definition 1.2]{LinsI}}]\label{bivfunc}
     The bivariate representation and the bivariate conjugacy class zeta functions of $\mathbf{G}(\mathcal{O})$ are
%
\begin{align*}
\mathcal{Z}_{\mathbf{G}(\mathcal{O})}^{\mathrm{irr}}(s_{1},s_{2})=\sum_{(0)\neq I\trianglelefteq\mathcal{O}}\zeta_{\mathbf{G}(\mathcal{O}/I)}^{\mathrm{irr}}(s_{1})|\mathcal{O}:I|^{-s_{2}},
\\
\mathcal{Z}_{\mathbf{G}(\mathcal{O})}^{\mathrm{cc}}(s_{1},s_{2})=\sum_{(0)\neq I\trianglelefteq\mathcal{O}}\zeta_{\mathbf{G}(\mathcal{O}/I)}^{\mathrm{cc}}(s_{1})|\mathcal{O}:I|^{-s_{2}}
\end{align*}
%
respectively, where $s_{1}$ and $s_{2}$ are complex variables.
\end{define}
%
Note that this definition uses the regular representation zeta function, rather than the one defined for twist equivalence. Lins showed that these converge for sufficiently large $s_i$, and for any $\mathcal T$-group $\mathbf G(\mathcal O)$~\cite[Proposition 2.3]{LinsI}. 

Denote by $\mathcal O_\mathfrak p$ the completion of $\mathcal O$ at a non-zero prime ideal $\mathfrak p$. By \cite[Proposition 2.2]{zeta1}, we can decompose the representation zeta function of $\mathbf{G}(\mathcal O)$ into an Euler product on its $\mathfrak p$-local factors, where $\mathfrak p$ ranges over the spectrum of $\mathcal O$,
\begin{align}\label{euler}
\zeta^{\widetilde{\mathrm{irr}}}_{\mathbf{G}(\mathcal O)}(s) = \prod_{\mathfrak p}\zeta^{\widetilde{\mathrm{irr}}}_{\mathbf{G}(\mathcal O_\mathfrak p)}(s).
\end{align}
%
Most results in this area rely on reducing our study to the local factors and extending them globally via this relation. By \cite[Proposition 2.4]{LinsI}, we can extend this factorisation to the bivariate zeta functions
\begin{align}\label{biveuler}
\mathcal Z_{\mathbf{G}(\mathcal{O})}^{\ast}(s_{1},s_{2}) = \prod_{\mathfrak{p} }\mathcal Z_{\mathbf{G}(\mathcal{O}_{\mathfrak{p}})}^{\ast}(s_{1},s_{2}),\end{align}
where $\ast\in\{\mathrm{irr,cc}\}$. This allows us to consider the local factors of bivariate functions and relate them to their univariate counterparts.
%
\subsubsection{$\mathbb Z$-Lie lattices}
Much current research (for example \cite{z,LinsI,LinsII}) focuses on three infinite families of $\mathcal T$-groups of nilpotency class 2, introduced by Stasinski and Voll in \cite{zeta1}. These arise from the following $\mathbb Z$-Lie lattices.

\begin{define}\label{famililes}
For $n \in \mathbb{N}$ and $\delta \in \{0,1\}$, we define the following $\mathbb{Z}$-Lie lattices
\begin{align*}
&\mathcal{F}_{n,\delta} = \langle x_{k}, y_{ij} \mid [x_{i}, x_{j}] = y_{ij}, \ 1 \leq k \leq 2n + \delta, \ 1 \leq i < j \leq 2n + \delta \rangle,\\
&\mathcal{G}_{n} = \langle x_{k}, y_{ij} \mid [x_{i}, x_{n+j}] = y_{ij}, \ 1 \leq k \leq 2n, \ 1 \leq i, j \leq n \rangle,\\
&\mathcal{H}_{n} = \langle x_{k}, y_{ij} \mid [x_{i}, x_{n+j}] = y_{ij}, \ [x_{j}, x_{n+i}] = y_{ij}, \ 1 \leq k \leq 2n, \ 1 \leq i \leq j \leq n \rangle
\end{align*}
where by convention any brackets not in the presentation are trivial.
\end{define}

By Section \ref{sec2}, these families have an associated family of group schemes from which we can obtain families of class 2 nilpotent $\mathcal T$-groups. These have been constructed in such a way that their Lie rings mirror the prehomogenous vector spaces $\mathrm{Alt}_{2n}(\mathbb C)$, $\mathrm{Mat}_{n}(\mathbb C)$, and $\mathrm{Sym}_{n}(\mathbb C)$ respectively. Stasinski and Voll show a resemblance in \cite[Theorem B]{zeta1} between the local representation zeta functions of the $\mathcal T$-groups obtained from the families \ref{famililes}, and the \textit{Igusa zeta functions} associated with the prehomogenous vector spaces. These Igusa zeta functions are given as $\mathfrak p$-adic integrals and are well known and studied, providing key insights into the structure of the representation zeta functions. This is further explored in \cite[Section 6]{zeta1}.
\subsubsection{Applications of bivariate zeta functions}\label{applications}
As discussed in \cite{LinsI}, bivariate zeta functions can be used to obtain results relating to regular zeta functions via specializations. We provide two examples of this.

Let $k(G)$ denote the \textit{class number} of some finite group $G$, i.e.~the number of conjugacy classes of $G$, or equivalently the number of irreducible representations over $\mathbb C$ (up to isomorphism). The \textit{class number zeta function}, for a $\mathcal T$-group $\mathbf G(\mathcal O)$, is defined to be
%
\begin{align}\label{classnumber}
\zeta_{\mathbf{G}(\mathcal O)}^\mathrm{k}(s) = \sum_{(0) \neq I \trianglelefteq\mathcal O}\mathrm{k}(\mathbf{G}(\mathcal O/I))|\mathcal O:I|^{-s}.\end{align}
%
This function is often explored alongside the representation and conjugacy class zeta functions, and often in the context of ask (\textit{average sized kernel}) zeta functions (cf. \cite{rossmann2021groupsgraphshypergraphsaverage}). They were introduced for $p$-adic linear groups by du Sautoy in \cite{DUSAUTOY_2005}.
%
Directly from Definition \ref{bivfunc}, we get

\begin{align}\label{0second}\mathcal Z^{\mathrm{irr}}_{\mathbf{G}(\mathcal O)}(0,s)=\mathcal Z^{\mathrm{cc}}_{\mathbf{G}(\mathcal O)}(0,s)=\zeta_{\mathbf{G}(\mathcal O)}^\mathrm{k}(s),\end{align}
%
allowing us to specialize results onto this class function. For example, \cite[Theorem 1.4]{LinsI} can be applied to give rational functions for almost all local factors of $\zeta_{\mathbf{G}(\mathcal O)}^\mathrm{k}(s)$.

 For $\mathcal T_2$-groups we can recover the local factors of the univariate representation zeta function via the following proposition. Recall that $\mathcal O_\mathfrak p$ is the completion of $\mathcal O$ at a fixed non-zero prime ideal $\mathfrak p$. Let $\mathbf G(\mathcal O)$ be a $\mathcal T_2$-group arising from a Lie lattice $\Lambda$, and let $\mathfrak g= \Lambda \otimes_{\mathcal O}\mathcal O_\mathfrak p$ (for context, see Section \ref{basisconstruct}). We denote by $r$ the rank of $\mathfrak g/\mathfrak g'$ (where $\mathfrak g'$ is the derived Lie sublattice), and by $q$ the cardinality of the residue field, $|\mathcal O/\mathfrak p|$. 

\begin{prop}[{\cite[Proposition 4.11]{LinsI}}]\label{zetabivequiv} 
Let $\mathbf{G}(\mathcal O_\mathfrak p)$ be a $\mathcal T$-group of nilpotency class 2. Then,
$$\zeta^{\widetilde{\mathrm{irr}}}_{\mathbf G(\mathcal O_\mathfrak p)}(s) = (1-q^{r-s_2})\mathcal Z^\mathrm{irr}_{\mathbf{G}(\mathcal O_\mathfrak p)}(s_1,s_2)\bigg|_{\substack{s_1 \to s - 2 \\ s_2 \to r}},$$
given that both sides converge.\end{prop}
\begin{remark}
     This specialization does not hold in general for nilpotency class greater than $2$, for a counterexample of class $3$ see \cite[Section 4.3]{LinsI}.
\end{remark}
%
These specialisations can be used to recover rational functions for the class number and representation zeta function, given a rational function for the bivariate zeta functions (cf. \cite[\emph{Example 1.9}]{LinsI}).

We can apply Proposition~\ref{zetabivequiv} to our result \eqref{kfoldirr} for $k$-fold central products (noting that this result holds for local factors, cf. Section \ref{zirr}) to obtain an equivalence for the bivariate representation zeta function,
%
$$(1-q^{r-s_2})\mathcal Z^\mathrm{irr}_{\mathbf G(\mathcal O_\mathfrak p)}(s_1,s_2)\bigg|_{\substack{s_1 \to ks - 2 \\ s_2 \to r}}=(1-q^{r-s_2})\mathcal Z^\mathrm{irr}_{\times^k_Z\mathbf G(\mathcal O_\mathfrak p)}(s_1,s_2)\bigg|_{\substack{s_1 \to s - 2 \\ s_2 \to r}}.$$
%
\subsubsection{Context and techniques} There are several techniques used to approach and compute zeta functions. We briefly introduce the ones covered in this paper. The \textit{Kirillov orbit method} was originally introduced by Kirillov~\cite{Kirillov_1962}, but adapted for $\mathcal T$-groups by Howe~\cite{Howe1977ONRO}, and gives a correspondence between the irreducible representations of $\mathbf G(\mathcal O_\mathfrak p)$ and the finite co-adjoint orbits in the dual of the associated Lie lattice $\mathfrak g$. We currently have sufficient but not necessary conditions on when we can apply this (specifically on the characteristic of the residue field $\mathcal O/\mathfrak p$), but it is shown in \cite[Section 2.4]{zeta1} that we can apply it to any $\mathcal T_2$-group. Under these assumptions, the zeta functions can be expressed in terms of a Poincaré series, which can in turn be reduced to the computation of $p$-adic integrals associated with certain polynomials determined by the structure of $\mathfrak g$~\cite{voll2008}.

\subsection{Main results}
Throughout this paper we restrict our attention to $\mathcal T_2$-groups, for which certain methods simplify and allow us to discern stronger results.

It is known that, for a $\mathcal T_2$-group G,
\begin{align}\label{kfoldirr}
\zeta^{\widetilde{\mathrm{irr}}}_{\times_{Z}^{k}G}(s) = \zeta^{\widetilde{\mathrm{irr}}}_{G}(ks).\end{align}
%
In Section \ref{zirr} we provide a detailed and accessible proof for this fact, which we extend to the bivariate representation zeta function.

%
% $$(1-q^{r-s_2})\mathcal Z^\mathrm{irr}_{G(\mathcal O)}(s_1,s_2)\bigg|_{\substack{s_1 \to ks - 2 \\ s_2 \to r}}=(1-q^{r-s_2})\mathcal Z^\mathrm{irr}_{\times^k_ZG(\mathcal O)}(s_1,s_2)\bigg|_{\substack{s_1 \to s - 2 \\ s_2 \to r}}.$$
%
Snocken used \eqref{kfoldirr} in \cite[Theorem 4.22]{snocken} to prove that for any rational $\alpha \in \mathbb Q$, we can find a $\mathcal T$-group (more specifically one of the form $G_{m,n}$, cf. Definition \ref{Gmn}) such that the abscissa of convergence of its representation zeta function is $\alpha$. This suggests that studying these functions for central products may lead to meaningful insights into the original groups.

In the second half of this paper we will focus on $k$-fold central products of groups arising from the family of Lie lattices $\mathcal G_{m,n}$, which are in turn a generalisation of the family $\mathcal G_n$ in Definition \ref{famililes}.
\begin{define}\label{Gmn}
    Let $m,n \in \mathbb N$. Define the following $\mathbb Z$-Lie lattice
    $$\mathcal G_{m,n}= \langle x_{1},\dots,x_{m+n}, z_{ij} \mid [x_{i}, x_{m+j}] = z_{ij}, \ 1 \leq i \leq m, \ 1 \leq j \leq n \rangle.$$
    Note that in the case $m=n$ we obtain the original family $\mathcal G_n$. Throughout, let $G_{m,n}$ denote the $\mathcal T$-groups associated with the family $\mathcal G_{m,n}$, and $\mathbf G_{m,n}$ be the associated group scheme so that $\mathbf G_{m,n}(\mathcal O)=G_{m,n}$.
\end{define}

In \cite{z}, Zordan proved the following result for the bivariate conjugacy class zeta function of $G_{m,n}$.

\begin{theorem}[{\cite[Theorem C]{z}}]\label{Zordanstheorem} Let $\mathfrak p$ be a non-zero prime ideal of $\mathcal O$ and let q denote the cardinality of its residue field. Then,
$$\mathcal{Z}^{\text{cc}}_{\mathbf G_{m,n}(\mathcal{O}_\mathfrak{p})}(s_1,s_2)=\frac{N^{\text{cc}}_{m,n;q}(q^{-s_1},q^{-s_2})}{D^{\text{cc}}_{m,n;q}(q^{-s_1},q^{-s_2})},$$ where
\begin{align*}
    N^{\text{cc}}_{m,n;q}(T_1,T_2)&= T_1q^{m+n} 
\\&+\ q^{2mn}(q^n+q^m-q^{m+n})\ T_1^{m+n+1}T_2^2
\\&+\ q^{mn+1}(1-q^m-q^n)\ T_1^{m+n}T_2
\\&+\ q^{2mn+1}(q^nT_1^{2m+n}+q^mT_1^{2n+m})\ T_2^2
\\&-\ q^{mn}(q^nT_1^{m+1}+q^mT_1^{n+1})T_2
\\&-\ T_1^{2(m+n)}T_2^3q^{3mn+1}
    \\    D^{\text{cc}}_{m,n;q}(T_1,T_2)=&\ T_1 q^{m+n} (1 - T_1^{m+n-1} T_2 q^{mn+1}) (1 - T_1^n T_2 q^{(m-1)(n+1)+1}) \\& \cdot (1 - T_1^m T_2 q^{(n-1)(m+1)+1}) (1 - T_2 q^{mn}).
    \end{align*}
    
\end{theorem}

In Section \ref{zcc} we extend this theorem for $k$-fold central products of $G_{m,n}$ in order to prove the following new result.
\begin{theorem}\label{BigTheorem}
Let $\mathfrak p$ be a non-zero prime ideal of $\mathcal O$ and let q denote the cardinality of its residue field. Let $\times_Z^k G_{m,n}$ denote the $k$-fold central product of $G_{m,n}$. Then,
    $$\mathcal{Z}^{\text{cc}}_{\times ^k_Z\mathbf G_{m,n}(\mathcal{O}_\mathfrak{p})}(s_1,s_2)=\frac{N^{\text{cc}}_{m,n;q}(q^{-s_1},q^{-s_2})}{D^{\text{cc}}_{m,n;q}(q^{-s_1},q^{-s_2})},$$ where
%     \begin{align*}
%     N^{\text{cc}}_{m,n;q}(T_1,T_2)=& \ T_1q^{m+n}q^{(k-1)(m+n)} 
% \\&+\ q^{2mn}(q^{kn}+q^{km}-q^{k(m+n)})\ T_1^{m+n+1}T_2^2q^{(k-1)(m+n)} 
% \\&+\ q^{mn+1}(1-q^{km}-q^{kn})\ T_1^{m+n}T_2q^{(k-1)(m+n)} 
% \\&+\ q^{2mn+1}(q^nT_1^{2m+n}+q^mT_1^{2n+m})\ T_2^2q^{2(k-1)(m+n)} 
% \\&-\ q^{mn}(q^nT_1^{m+1}+q^mT_1^{n+1})T_2q^{(k-1)(m+n)} 
% \\&-\ T_1^{2(m+n)}T_2^3q^{3mn+1}q^{2(k-1)(m+n)} 
%     \\    D^{\text{cc}}_{m,n;q}(T_1,T_2)=&\ T_1 q^{m+n}q^{(k-1)(m+n)} (1 - T_1^{m+n-1} T_2 q^{mn+1}q^{(k-1)(m+n)}) \\&\cdot(1 - T_1^n T_2 q^{(m-1)(n+1)+1}q^{(k-1)(m+n)}) \\& \cdot (1 - T_1^m T_2 q^{(n-1)(m+1)+1}q^{(k-1)(m+n)}) (1 - T_2 q^{mn})
%     \end{align*}
% OR

    \begin{align*}
    N^{\text{cc}}_{m,n;q}(T_1,T_2)=& \ T_1q^{m+n}
\\&+\ q^{2mn}(q^{kn}+q^{km}-q^{k(m+n)})\ T_1^{m+n+1}T_2^2
\\&+\ q^{mn+1}(1-q^{km}-q^{kn})\ T_1^{m+n}T_2
\\&+\ q^{2mn+1}(q^nT_1^{2m+n}+q^mT_1^{2n+m})\ T_2^2q^{(k-1)(m+n)} 
\\&-\ q^{mn}(q^nT_1^{m+1}+q^mT_1^{n+1})T_2
\\&-\ T_1^{2(m+n)}T_2^3q^{3mn+1}q^{(k-1)(m+n)} 
    \\    D^{\text{cc}}_{m,n;q}(T_1,T_2)=&\ T_1 q^{m+n}(1 - T_1^{m+n-1} T_2 q^{mn+1}q^{(k-1)(m+n)}) \\&\cdot(1 - T_1^n T_2 q^{(m-1)(n+1)+1}q^{(k-1)(m+n)}) \\& \cdot (1 - T_1^m T_2 q^{(n-1)(m+1)+1}q^{(k-1)(m+n)}) (1 - T_2 q^{mn}).
    \end{align*}
    
\end{theorem}

This provides a direct formula for the bivariate conjugacy class zeta function of the $k$-fold central product of groups $G_{m,n}$. In particular it is not a direct scaling of the function for $G_{m,n}$, unlike the complementary representation zeta function, cf. Remark \ref{endremark}.
\newpage
\subsection{Notation} The following table summarises some frequently used notations.
\renewcommand{\arraystretch}{1.2}
\begin{table}[h]
    \centering
    
    \begin{tabular}{c|l|c}
        $\mathbb{N}$ & $\{1,2,\dots\}$\\
        $\mathbb{N}_0$ & $\{0,1,2,\dots\}$\\
        $[n]$& $\{1,2,\dots,n\}$\\ \hline
        $K$ & number field \\
        $\mathcal{O}$ & ring of integers of $K$ \\
        $\mathfrak{p}$ & non-zero prime ideal of $\mathcal{O}$ \\
        $\mathfrak o =\mathcal{O}_\mathfrak{p}$ & completion of $\mathcal{O}$ at $\mathfrak{p}$ \\
        $q$ & cardinality of $\mathcal O/\mathfrak p$\\
        \hline
        $\zeta_G^{\mathrm{cc}}(s)$&Conjugacy class zeta function&\ref{zetaorigdef}\\
        $\zeta_G^{\mathrm{irr}}(s)$&Representation zeta function&\ref{zetaorigdef}\\
        $\zeta_G^{\widetilde{\mathrm{irr}}}(s)$&Twist representation zeta function&\ref{twistzeta}\\
        $\zeta_G^{\mathrm{k}}(s)$&Class number zeta function&\ref{classnumber}\\
        \hline
        $\mathcal G_{m,n}$ & $\mathbb{Z}$-Lie lattice&\ref{Gmn}\\
        $G_{m,n}$ & Associated groups to $\mathcal G_{m,n}$&\ref{malcev}\\
        $\times_Z^kG$ & $k$-fold central product of $G$&\ref{k-fold}\\
        $\mathcal A(\boldsymbol{X})$ & $A$-commutator matrix&\ref{comms}\\
        $\mathcal B(\boldsymbol{X})$ & $B$-commutator matrix&\ref{comms}\\ \hline
        $M^\ast$ & $M\smallsetminus\mathfrak pM$&\ref{zirr}\\
        $W_d(\mathfrak o)$& $(\mathfrak o^d)^\ast$&\ref{zirr}\\
        $W_{d,N}(\mathfrak o)$&$((\mathfrak o/\mathfrak p^N)^d)^\ast$&\ref{zirr}\\ \hline
        $\nu_\mathfrak{p}$ & $\mathfrak{p}$-adic valuation \\
        $|\cdot|_\mathfrak{p}$ & $\mathfrak{p}$-adic norm \\
        $||\{x_i\}_{i \in I}||_\mathfrak{p}$ & max$_{i \in I}(|x_i|_\mathfrak{p})$ \\
    \end{tabular}
    \label{tab:my_label}
\end{table}
\renewcommand{\arraystretch}{1}
