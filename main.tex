\documentclass[11pt,a4paper]{amsart}
\usepackage{graphicx} % Required for inserting images
\usepackage{color, setspace, placeins, todonotes}
\usepackage{minted}
\usemintedstyle{tango}
\usepackage[hidelinks]{hyperref}
\usepackage{amsthm,amsmath,amssymb, mathtools, xfrac, scalerel, multicol,titlecaps,amsaddr}
\usepackage[utf8]{inputenc}
%\usepackage[style=ieee]{biblatex}
\usepackage[numbers, sort&compress]{natbib}
%\addbibresource{bib.bib}  
\usepackage{enumitem,booktabs} 
\usepackage[T1]{fontenc}
\usepackage{listings, fix-cm}

%\usepackage[margin=1in]{geometry}
\usepackage{lipsum}
%\setminted{fontsize=\small}

\numberwithin{equation}{subsection}
\setlength\intextsep{12pt}

\let\oldsection\section% Store \section
\renewcommand{\section}{% Update \section
  \renewcommand{\theequation}{\thesection.\arabic{equation}}% Update equation number
  \oldsection}% Regular \section
\let\oldsubsection\subsection% Store \subsection
\renewcommand{\subsection}{% Update \subsection
  \renewcommand{\theequation}{\thesubsection.\arabic{equation}}% Update equation number
  \oldsubsection}% Regular \subsection
\renewcommand\labelenumi{(\roman{enumi})}
\renewcommand\theenumi\labelenumi
\newcommand{\customsmall}{\fontsize{10.55pt}{11pt}\selectfont}

%testing%
\makeatletter
%Table of Contents
\setcounter{tocdepth}{3}

% Add bold to \section titles in ToC and remove . after numbers
\renewcommand{\tocsection}[3]{%
  \indentlabel{\@ifnotempty{#2}{\bfseries\ignorespaces#1 #2\quad}}\bfseries#3}
% Remove . after numbers in \subsection
\renewcommand{\tocsubsection}[3]{%
  \indentlabel{\@ifnotempty{#2}{\ignorespaces#1 #2\quad}}#3}
%\let\tocsubsubsection\tocsubsection% Update for \subsubsection
%...

% \newcommand\@dotsep{4.5}
% \def\@tocline#1#2#3#4#5#6#7{\relax
%   \ifnum #1>\c@tocdepth % then omit
%   \else
%     \par \addpenalty\@secpenalty\addvspace{#2}%
%     \begingroup \hyphenpenalty\@M
%     \@ifempty{#4}{%
%       \@tempdima\csname r@tocindent\number#1\endcsname\relax
%     }{%
%       \@tempdima#4\relax
%     }%
%     \parindent\z@ \leftskip#3\relax \advance\leftskip\@tempdima\relax
%     \rightskip\@pnumwidth plus1em \parfillskip-\@pnumwidth
%     #5\leavevmode\hskip-\@tempdima{#6}\nobreak
%     \leaders\hbox{$\m@th\mkern \@dotsep mu\hbox{.}\mkern \@dotsep mu$}\hfill
%     \nobreak
%     \hbox to\@pnumwidth{\@tocpagenum{\ifnum#1=1\bfseries\fi#7}}\par% <-- \bfseries for \section page
%     \nobreak
%     \endgroup
%   \fi}
% \AtBeginDocument{%
% \expandafter\renewcommand\csname r@tocindent0\endcsname{0pt}
% }
% \def\l@subsection{\@tocline{2}{0pt}{2.5pc}{5pc}{}}
% \makeatother




%testing%







\theoremstyle{plain}
\newtheorem{theorem}{Theorem}[section]
\newtheorem{prop}[theorem]{Proposition}
\newtheorem{lemma}[theorem]{Lemma}
\newtheorem{coroll}[theorem]{Corollary}
\theoremstyle{definition}
\newtheorem{define}[theorem]{Definition}
\newtheorem*{example}{Example}
\theoremstyle{remark}
\newtheorem{remark}{Remark}


\title{Zeta Functions of $k$-Fold Central Products}

\author{Leo Jones} 
%\\ \scriptsize{Imperial College London}}


\author{\small Supervised by Michele Zordan}
\address{\small{{Department of Mathematics, Imperial College London}}}
\date{9th June 2025,\\ \textit{CID}: 02036984}
% \makeatletter
% \renewcommand{\maketitle}{
%   \begin{center}
%     {\Large\bfseries \@title\par}
%     \vspace{2em}
%     {\large \@author\par}
%     {\normalsize Imperial College London\par} % Add university here
%   \end{center}
% }
% \makeatother
\raggedbottom
\begin{document}



\begin{abstract}
    We explore representation and conjugacy class zeta functions as well as their extended bivariate forms, introduced by Lins~\cite{LinsI}, in the context of $k$-fold central products for finitely generated, torsion-free groups of nilpotency class 2. We prove a direct relationship between the representation zeta functions of these groups and their central products. We then look at a family of these groups arising from the $\mathbb Z$-Lie lattices $\mathcal G_{m,n}$, and derive a novel and explicit formula for the bivariate conjugacy class zeta function associated with their central products.
\end{abstract}
\maketitle
\newpage
\tableofcontents
\newpage
\section{Introduction}\label{intro}


\subsection{Background and motivation}
We begin with a concise overview of the study of representation and conjugacy class growth, situating the present work within the broader context of current research in these areas. This is followed by a brief summary of the development of zeta functions with emphasis on those we cover in this paper. We extend this to introduce Euler products decompositions, bivariate zeta functions, and Lie lattices, which together form the framework necessary for stating our main results.


\subsubsection{Representation growth}
A representation of a group $G$ is a pair $(V, \rho)$ consisting of a vector space $V$ and group a homomorphism $\rho : G\to \mathrm{GL}_n(k)$. When $G$ is finite, all representations are semisimple and Maschke's theorem allows us to decompose them into irreducible representations. There are only finitely many of these, determined by their character (the trace function of $\rho$). For certain infinite groups there exist generalisations of Maschke's theorem, for example the Peter-Weyl theorem (cf. \cite[Theorem 4.1]{Bump2013LieGroups}) for compact Lie groups. In general, when $G$ is not finite it is more difficult to classify representations, and key areas of interest include non-compact Lie groups (via highest weight theory), and profinite groups (given by Galois groups of infinite degree field extensions, cf.~\cite{Waterhouse1974}). In this paper, we will consider the representation growth of $\mathcal T$-groups, which are finitely generated, nilpotent, and torsion-free.

\subsubsection{Conjugacy class growth}\label{conjdiscussion}
Conjugacy classes are equivalence classes partitioned by the relation "equal under conjugation" by elements of the group. Again, when $G$ is finite these are easily found and classified. In fact, the number of conjugacy classes is precisely the number of irreducible representations over $\mathbb C$ (up to isomorphism). For studying infinite groups, a key theorem to note is Neumann's theorem, which states that there exists an upper bound to all conjugacy class sizes if and only if the derived subgroup (the subgroup formed by commutators) is finite \cite[Theorem 3.1]{Neumann1954}.

In this paper we look at the growth of a counting function for conjugacy classes of a given cardinality, however other growth functions exist and are also key areas of interest. For example, let $G$ be a finitely generated group generated by a set $S$. We can define \textit{word length}, where elements of the group are represented by words formed from the alphabet $S\cup S^{-1}$. The \textit{Cayley graph} encodes certain structural properties of finitely generated groups, and we say an element lies in the ball of radius $n$ (of the Cayley graph, centred at the identity) if it has a representative of length $\leq n$. Common growth functions include counting the number of conjugacy classes that intersect the ball of radius $n$, as well as the number of elements within a given class that fall inside this ball. These have been proven to have polynomial growth for certain groups, see \cite{conjugacyclassgrowthvirtually} and \cite{rationalgrowthvirtuallyabelian}.

A remark in \cite{GrunewaldSegalSmith1988} posed the question of studying zeta functions of the conjugacy class-counting function, specifically for $\mathcal T$-groups. Modifications of this function exist for when our groups have infinite conjugacy classes of a given size; we will explore these later.


\subsubsection{Zeta functions}
Zeta functions of groups were introduced by \\Grunewald, Segal, and Smith~\cite{GrunewaldSegalSmith1988} in order to study subgroup growth of finitely generated groups. In their paper, zeta functions are given by a Dirichlet series
$$\zeta_G(s)=\sum_{H\leq G}|G:H|^{-s}.$$
%
More specifically they studied $\mathcal T$-groups, counting their finite index normal subgroups and conjugacy classes of subgroups. This was continued by Hrushovski et al.~\cite{handm} who studied representation zeta functions of $\mathcal T$-groups, counting irreducible representations up to (twist equivalence) of finite-dimensional complex characters.

An \textit{Euler product} is a decomposition of a Dirichlet series into an infinite product, indexed by primes. This was originally proven for the series 
%
$$\sum^{\infty}_{n=0}\frac{1}{n^s}= \prod_p\frac{1}{1-p^{-s}},$$
%
which later became known as the \textit{Riemann zeta function}, $\zeta(s)$. Most zeta functions satisfy these Euler products, including all the ones mentioned in this paper. By decomposing them into an Euler product, we can study their local properties. We have 
%
$$\zeta_G(s) = \prod_{p}\zeta_{G,p}(s),$$
%
for primes $p$, where $\zeta_{G,p}(s)$ sums over finite $p$-power index subgroups. 

Given a number field $K$ with a ring of integers $\mathcal O$, an $\mathcal O$-Lie lattices is a free and finitely generated $\mathcal O$-module equipped with a Lie bracket (a bilinear, antisymmetric operation satisfying the Jacobi identity). It was shown by Grunewald et al.~\cite{GrunewaldSegalSmith1988} that, given a $\mathcal T$-group $G$, we have an associated $\mathbb Z$-Lie lattice $L$, such that for almost all (all but finitely many) primes $p$, we have
%
\begin{align}\label{LIE}\zeta_{G,p}(s) = \zeta_{L,p}(s).\end{align}
%
We currently only have sufficient but not necessary conditions on the primes $p$ for which this holds. In \cite{voll2008}, Voll uses this to prove that local functional equations hold for almost all factors of representation and conjugacy class zeta functions of $\mathcal T$-groups. The relation \eqref{LIE} is a key result used in much subsequent research into $\mathcal T$-groups.

Let $G$ be a group, and denote by $r_n(G)$ for $n\in\mathbb N$, the number of isomorphism classes of $n$-dimensional irreducible representations. When $r_n(G)$ is finite we call $G$ \textit{representation rigid}, and we can encode it into a representation zeta function in order to study the growth of $r_n(G)$. Similarly, we define $c_n(G)$ to be the number of conjugacy classes of $G$ with cardinality $n$, and when this is finite $G$ is \textit{conjugacy rigid}. These functions are defined as follows.

\begin{define}\label{zetaorigdef}
    The representation zeta function and the conjugacy class zeta function for a group $G$ are, respectively
    \begin{align*}
        &\zeta^\mathrm{irr}_G(s)= \sum^\infty_{n=1}r_n(G)n^{-s} &\text{and}&&
        \zeta^\mathrm{cc}_G(s)= \sum^\infty_{n=1}c_n(G)n^{-s}.
    \end{align*}
\end{define}

Recall our focus on $\mathcal T$-groups, which are finitely generated, torsion-free, and nilpotent. They are neither representation rigid nor conjugacy rigid, hence we need to alter the zeta functions in order to study them.

In \cite{handm}, Hrushovski et al. introduced \textit{twist equivalent} representations, classes of which $\mathcal T$-groups have finitely many. These differ by tensoring with a one-dimensional representation of $G$ ie. two representations $\rho$ and $\sigma$ are twist equivalent if $\rho \cong \chi \otimes\sigma$ where $\chi$ is a one dimensional (and therefore irreducible) representation of $G$. Under this equivalence, we call classes of representations \textit{twist iso-classes}. We denote by $\widetilde{r}_n(G)$ the number of twist iso-classes of dimension $n$ in $G$. The updated zeta function is
%
\begin{align}\label{twistzeta}  
\zeta_G^{\widetilde{\mathrm{irr}}}(s) = \sum^\infty_{n=1} \widetilde{r}_n(G)n^{-s}.\end{align}
%
The fact that $\widetilde{r}_n(G)$ grows polynomially follows from the PSG (polynomial subgroup growth) Theorem \cite[Theorem A]{LubotzkyMann1987}, and \cite[Lemma 2.1]{zeta1}, hence this sum converges on a right half of $\mathbb C$. It is interesting to note that twist equivalence is also studied for conjugacy growth, where two elements $x,y \in G$ are \textit{twist conjugate} if $y=g\cdot x \cdot \phi(g^{-1})$, with $g \in G $ and $\phi \in \mathrm{Aut}(G)$ (cf.~\cite{twistconj,dekimpe2024twistedconjugacygrowthvirtually}), however this is not covered in this paper.

% \textcolor{red}{here!}rational in \cite{rnrational}.
As before, let $K$ be a number field and $\mathcal O$ its ring of integers. Then we can obtain $\mathcal{T}$-groups from unipotent group schemes $\mathbf G$ over $\mathcal O$ (cf. Section \ref{sec2}). All the $\mathcal T$-groups considered in this paper -- and moreover, all $\mathcal T$-groups of nilpotency class 2, which we will call $\mathcal T_2$-groups -- can be constructed in this way. Throughout, we shall refer to $\mathcal T$-groups by their form $\mathbf G(\mathcal O)$.

An extension of the zeta functions \eqref{zetaorigdef} are the bivariate zeta functions introduced by Lins in \cite{LinsI}. To overcome the issue of rigidity, we look at $r_n(Q)$ and $c_n(Q)$ for principle congruence quotients $Q$ of $\mathbf G (Q)$, and define the following. 
%
\begin{define}[{\cite[Definition 1.2]{LinsI}}]\label{bivfunc}
     The bivariate representation and the bivariate conjugacy class zeta functions of $\mathbf{G}(\mathcal{O})$ are
%
\begin{align*}
\mathcal{Z}_{\mathbf{G}(\mathcal{O})}^{\mathrm{irr}}(s_{1},s_{2})=\sum_{(0)\neq I\trianglelefteq\mathcal{O}}\zeta_{\mathbf{G}(\mathcal{O}/I)}^{\mathrm{irr}}(s_{1})|\mathcal{O}:I|^{-s_{2}},
\\
\mathcal{Z}_{\mathbf{G}(\mathcal{O})}^{\mathrm{cc}}(s_{1},s_{2})=\sum_{(0)\neq I\trianglelefteq\mathcal{O}}\zeta_{\mathbf{G}(\mathcal{O}/I)}^{\mathrm{cc}}(s_{1})|\mathcal{O}:I|^{-s_{2}}
\end{align*}
%
respectively, where $s_{1}$ and $s_{2}$ are complex variables.
\end{define}
%
Note that this definition uses the regular representation zeta function, rather than the one defined for twist equivalence. Lins showed that these converge for sufficiently large $s_i$, and for any $\mathcal T$-group $\mathbf G(\mathcal O)$~\cite[Proposition 2.3]{LinsI}. 

Denote by $\mathcal O_\mathfrak p$ the completion of $\mathcal O$ at a non-zero prime ideal $\mathfrak p$. By \cite[Proposition 2.2]{zeta1}, we can decompose the representation zeta function of $\mathbf{G}(\mathcal O)$ into an Euler product on its $\mathfrak p$-local factors, where $\mathfrak p$ ranges over the spectrum of $\mathcal O$,
\begin{align}\label{euler}
\zeta^{\widetilde{\mathrm{irr}}}_{\mathbf{G}(\mathcal O)}(s) = \prod_{\mathfrak p}\zeta^{\widetilde{\mathrm{irr}}}_{\mathbf{G}(\mathcal O_\mathfrak p)}(s).
\end{align}
%
Most results in this area rely on reducing our study to the local factors and extending them globally via this relation. By \cite[Proposition 2.4]{LinsI}, we can extend this factorisation to the bivariate zeta functions
\begin{align}\label{biveuler}
\mathcal Z_{\mathbf{G}(\mathcal{O})}^{\ast}(s_{1},s_{2}) = \prod_{\mathfrak{p} }\mathcal Z_{\mathbf{G}(\mathcal{O}_{\mathfrak{p}})}^{\ast}(s_{1},s_{2}),\end{align}
where $\ast\in\{\mathrm{irr,cc}\}$. This allows us to consider the local factors of bivariate functions and relate them to their univariate counterparts.
%
\subsubsection{$\mathbb Z$-Lie lattices}
Much current research (for example \cite{z,LinsI,LinsII}) focuses on three infinite families of $\mathcal T$-groups of nilpotency class 2, introduced by Stasinski and Voll in \cite{zeta1}. These arise from the following $\mathbb Z$-Lie lattices.

\begin{define}\label{famililes}
For $n \in \mathbb{N}$ and $\delta \in \{0,1\}$, we define the following $\mathbb{Z}$-Lie lattices
\begin{align*}
&\mathcal{F}_{n,\delta} = \langle x_{k}, y_{ij} \mid [x_{i}, x_{j}] = y_{ij}, \ 1 \leq k \leq 2n + \delta, \ 1 \leq i < j \leq 2n + \delta \rangle,\\
&\mathcal{G}_{n} = \langle x_{k}, y_{ij} \mid [x_{i}, x_{n+j}] = y_{ij}, \ 1 \leq k \leq 2n, \ 1 \leq i, j \leq n \rangle,\\
&\mathcal{H}_{n} = \langle x_{k}, y_{ij} \mid [x_{i}, x_{n+j}] = y_{ij}, \ [x_{j}, x_{n+i}] = y_{ij}, \ 1 \leq k \leq 2n, \ 1 \leq i \leq j \leq n \rangle
\end{align*}
where by convention any brackets not in the presentation are trivial.
\end{define}

By Section \ref{sec2}, these families have an associated family of group schemes from which we can obtain families of class 2 nilpotent $\mathcal T$-groups. These have been constructed in such a way that their Lie rings mirror the prehomogenous vector spaces $\mathrm{Alt}_{2n}(\mathbb C)$, $\mathrm{Mat}_{n}(\mathbb C)$, and $\mathrm{Sym}_{n}(\mathbb C)$ respectively. Stasinski and Voll show a resemblance in \cite[Theorem B]{zeta1} between the local representation zeta functions of the $\mathcal T$-groups obtained from the families \ref{famililes}, and the \textit{Igusa zeta functions} associated with the prehomogenous vector spaces. These Igusa zeta functions are given as $\mathfrak p$-adic integrals and are well known and studied, providing key insights into the structure of the representation zeta functions. This is further explored in \cite[Section 6]{zeta1}.
\subsubsection{Applications of bivariate zeta functions}\label{applications}
As discussed in \cite{LinsI}, bivariate zeta functions can be used to obtain results relating to regular zeta functions via specializations. We provide two examples of this.

Let $k(G)$ denote the \textit{class number} of some finite group $G$, i.e.~the number of conjugacy classes of $G$, or equivalently the number of irreducible representations over $\mathbb C$ (up to isomorphism). The \textit{class number zeta function}, for a $\mathcal T$-group $\mathbf G(\mathcal O)$, is defined to be
%
\begin{align}\label{classnumber}
\zeta_{\mathbf{G}(\mathcal O)}^\mathrm{k}(s) = \sum_{(0) \neq I \trianglelefteq\mathcal O}\mathrm{k}(\mathbf{G}(\mathcal O/I))|\mathcal O:I|^{-s}.\end{align}
%
This function is often explored alongside the representation and conjugacy class zeta functions, and often in the context of ask (\textit{average sized kernel}) zeta functions (cf. \cite{rossmann2021groupsgraphshypergraphsaverage}). They were introduced for $p$-adic linear groups by du Sautoy in \cite{DUSAUTOY_2005}.
%
Directly from Definition \ref{bivfunc}, we get

\begin{align}\label{0second}\mathcal Z^{\mathrm{irr}}_{\mathbf{G}(\mathcal O)}(0,s)=\mathcal Z^{\mathrm{cc}}_{\mathbf{G}(\mathcal O)}(0,s)=\zeta_{\mathbf{G}(\mathcal O)}^\mathrm{k}(s),\end{align}
%
allowing us to specialize results onto this class function. For example, \cite[Theorem 1.4]{LinsI} can be applied to give rational functions for almost all local factors of $\zeta_{\mathbf{G}(\mathcal O)}^\mathrm{k}(s)$.

 For $\mathcal T_2$-groups we can recover the local factors of the univariate representation zeta function via the following proposition. Recall that $\mathcal O_\mathfrak p$ is the completion of $\mathcal O$ at a fixed non-zero prime ideal $\mathfrak p$. Let $\mathbf G(\mathcal O)$ be a $\mathcal T_2$-group arising from a Lie lattice $\Lambda$, and let $\mathfrak g= \Lambda \otimes_{\mathcal O}\mathcal O_\mathfrak p$ (for context, see Section \ref{basisconstruct}). We denote by $r$ the rank of $\mathfrak g/\mathfrak g'$ (where $\mathfrak g'$ is the derived Lie sublattice), and by $q$ the cardinality of the residue field, $|\mathcal O/\mathfrak p|$. 

\begin{prop}[{\cite[Proposition 4.11]{LinsI}}]\label{zetabivequiv} 
Let $\mathbf{G}(\mathcal O_\mathfrak p)$ be a $\mathcal T$-group of nilpotency class 2. Then,
$$\zeta^{\widetilde{\mathrm{irr}}}_{\mathbf G(\mathcal O_\mathfrak p)}(s) = (1-q^{r-s_2})\mathcal Z^\mathrm{irr}_{\mathbf{G}(\mathcal O_\mathfrak p)}(s_1,s_2)\bigg|_{\substack{s_1 \to s - 2 \\ s_2 \to r}},$$
given that both sides converge.\end{prop}
\begin{remark}
     This specialization does not hold in general for nilpotency class greater than $2$, for a counterexample of class $3$ see \cite[Section 4.3]{LinsI}.
\end{remark}
%
These specialisations can be used to recover rational functions for the class number and representation zeta function, given a rational function for the bivariate zeta functions (cf. \cite[\emph{Example 1.9}]{LinsI}).

We can apply Proposition~\ref{zetabivequiv} to our result \eqref{kfoldirr} for $k$-fold central products (noting that this result holds for local factors, cf. Section \ref{zirr}) to obtain an equivalence for the bivariate representation zeta function,
%
$$(1-q^{r-s_2})\mathcal Z^\mathrm{irr}_{\mathbf G(\mathcal O_\mathfrak p)}(s_1,s_2)\bigg|_{\substack{s_1 \to ks - 2 \\ s_2 \to r}}=(1-q^{r-s_2})\mathcal Z^\mathrm{irr}_{\times^k_Z\mathbf G(\mathcal O_\mathfrak p)}(s_1,s_2)\bigg|_{\substack{s_1 \to s - 2 \\ s_2 \to r}}.$$
%
\subsubsection{Context and techniques} There are several techniques used to approach and compute zeta functions. We briefly introduce the ones covered in this paper. The \textit{Kirillov orbit method} was originally introduced by Kirillov~\cite{Kirillov_1962}, but adapted for $\mathcal T$-groups by Howe~\cite{Howe1977ONRO}, and gives a correspondence between the irreducible representations of $\mathbf G(\mathcal O_\mathfrak p)$ and the finite co-adjoint orbits in the dual of the associated Lie lattice $\mathfrak g$. We currently have sufficient but not necessary conditions on when we can apply this (specifically on the characteristic of the residue field $\mathcal O/\mathfrak p$), but it is shown in \cite[Section 2.4]{zeta1} that we can apply it to any $\mathcal T_2$-group. Under these assumptions, the zeta functions can be expressed in terms of a Poincaré series, which can in turn be reduced to the computation of $p$-adic integrals associated with certain polynomials determined by the structure of $\mathfrak g$~\cite{voll2008}.

\subsection{Main results}
Throughout this paper we restrict our attention to $\mathcal T_2$-groups, for which certain methods simplify and allow us to discern stronger results.

It is known that, for a $\mathcal T_2$-group G,
\begin{align}\label{kfoldirr}
\zeta^{\widetilde{\mathrm{irr}}}_{\times_{Z}^{k}G}(s) = \zeta^{\widetilde{\mathrm{irr}}}_{G}(ks).\end{align}
%
In Section \ref{zirr} we provide a detailed and accessible proof for this fact, which we extend to the bivariate representation zeta function.

%
% $$(1-q^{r-s_2})\mathcal Z^\mathrm{irr}_{G(\mathcal O)}(s_1,s_2)\bigg|_{\substack{s_1 \to ks - 2 \\ s_2 \to r}}=(1-q^{r-s_2})\mathcal Z^\mathrm{irr}_{\times^k_ZG(\mathcal O)}(s_1,s_2)\bigg|_{\substack{s_1 \to s - 2 \\ s_2 \to r}}.$$
%
Snocken used \eqref{kfoldirr} in \cite[Theorem 4.22]{snocken} to prove that for any rational $\alpha \in \mathbb Q$, we can find a $\mathcal T$-group (more specifically one of the form $G_{m,n}$, cf. Definition \ref{Gmn}) such that the abscissa of convergence of its representation zeta function is $\alpha$. This suggests that studying these functions for central products may lead to meaningful insights into the original groups.

In the second half of this paper we will focus on $k$-fold central products of groups arising from the family of Lie lattices $\mathcal G_{m,n}$, which are in turn a generalisation of the family $\mathcal G_n$ in Definition \ref{famililes}.
\begin{define}\label{Gmn}
    Let $m,n \in \mathbb N$. Define the following $\mathbb Z$-Lie lattice
    $$\mathcal G_{m,n}= \langle x_{1},\dots,x_{m+n}, z_{ij} \mid [x_{i}, x_{m+j}] = z_{ij}, \ 1 \leq i \leq m, \ 1 \leq j \leq n \rangle.$$
    Note that in the case $m=n$ we obtain the original family $\mathcal G_n$. Throughout, let $G_{m,n}$ denote the $\mathcal T$-groups associated with the family $\mathcal G_{m,n}$, and $\mathbf G_{m,n}$ be the associated group scheme so that $\mathbf G_{m,n}(\mathcal O)=G_{m,n}$.
\end{define}

In \cite{z}, Zordan proved the following result for the bivariate conjugacy class zeta function of $G_{m,n}$.

\begin{theorem}[{\cite[Theorem C]{z}}]\label{Zordanstheorem} Let $\mathfrak p$ be a non-zero prime ideal of $\mathcal O$ and let q denote the cardinality of its residue field. Then,
$$\mathcal{Z}^{\text{cc}}_{\mathbf G_{m,n}(\mathcal{O}_\mathfrak{p})}(s_1,s_2)=\frac{N^{\text{cc}}_{m,n;q}(q^{-s_1},q^{-s_2})}{D^{\text{cc}}_{m,n;q}(q^{-s_1},q^{-s_2})},$$ where
\begin{align*}
    N^{\text{cc}}_{m,n;q}(T_1,T_2)&= T_1q^{m+n} 
\\&+\ q^{2mn}(q^n+q^m-q^{m+n})\ T_1^{m+n+1}T_2^2
\\&+\ q^{mn+1}(1-q^m-q^n)\ T_1^{m+n}T_2
\\&+\ q^{2mn+1}(q^nT_1^{2m+n}+q^mT_1^{2n+m})\ T_2^2
\\&-\ q^{mn}(q^nT_1^{m+1}+q^mT_1^{n+1})T_2
\\&-\ T_1^{2(m+n)}T_2^3q^{3mn+1}
    \\    D^{\text{cc}}_{m,n;q}(T_1,T_2)=&\ T_1 q^{m+n} (1 - T_1^{m+n-1} T_2 q^{mn+1}) (1 - T_1^n T_2 q^{(m-1)(n+1)+1}) \\& \cdot (1 - T_1^m T_2 q^{(n-1)(m+1)+1}) (1 - T_2 q^{mn}).
    \end{align*}
    
\end{theorem}

In Section \ref{zcc} we extend this theorem for $k$-fold central products of $G_{m,n}$ in order to prove the following new result.
\begin{theorem}\label{BigTheorem}
Let $\mathfrak p$ be a non-zero prime ideal of $\mathcal O$ and let q denote the cardinality of its residue field. Let $\times_Z^k G_{m,n}$ denote the $k$-fold central product of $G_{m,n}$. Then,
    $$\mathcal{Z}^{\text{cc}}_{\times ^k_Z\mathbf G_{m,n}(\mathcal{O}_\mathfrak{p})}(s_1,s_2)=\frac{N^{\text{cc}}_{m,n;q}(q^{-s_1},q^{-s_2})}{D^{\text{cc}}_{m,n;q}(q^{-s_1},q^{-s_2})},$$ where
%     \begin{align*}
%     N^{\text{cc}}_{m,n;q}(T_1,T_2)=& \ T_1q^{m+n}q^{(k-1)(m+n)} 
% \\&+\ q^{2mn}(q^{kn}+q^{km}-q^{k(m+n)})\ T_1^{m+n+1}T_2^2q^{(k-1)(m+n)} 
% \\&+\ q^{mn+1}(1-q^{km}-q^{kn})\ T_1^{m+n}T_2q^{(k-1)(m+n)} 
% \\&+\ q^{2mn+1}(q^nT_1^{2m+n}+q^mT_1^{2n+m})\ T_2^2q^{2(k-1)(m+n)} 
% \\&-\ q^{mn}(q^nT_1^{m+1}+q^mT_1^{n+1})T_2q^{(k-1)(m+n)} 
% \\&-\ T_1^{2(m+n)}T_2^3q^{3mn+1}q^{2(k-1)(m+n)} 
%     \\    D^{\text{cc}}_{m,n;q}(T_1,T_2)=&\ T_1 q^{m+n}q^{(k-1)(m+n)} (1 - T_1^{m+n-1} T_2 q^{mn+1}q^{(k-1)(m+n)}) \\&\cdot(1 - T_1^n T_2 q^{(m-1)(n+1)+1}q^{(k-1)(m+n)}) \\& \cdot (1 - T_1^m T_2 q^{(n-1)(m+1)+1}q^{(k-1)(m+n)}) (1 - T_2 q^{mn})
%     \end{align*}
% OR

    \begin{align*}
    N^{\text{cc}}_{m,n;q}(T_1,T_2)=& \ T_1q^{m+n}
\\&+\ q^{2mn}(q^{kn}+q^{km}-q^{k(m+n)})\ T_1^{m+n+1}T_2^2
\\&+\ q^{mn+1}(1-q^{km}-q^{kn})\ T_1^{m+n}T_2
\\&+\ q^{2mn+1}(q^nT_1^{2m+n}+q^mT_1^{2n+m})\ T_2^2q^{(k-1)(m+n)} 
\\&-\ q^{mn}(q^nT_1^{m+1}+q^mT_1^{n+1})T_2
\\&-\ T_1^{2(m+n)}T_2^3q^{3mn+1}q^{(k-1)(m+n)} 
    \\    D^{\text{cc}}_{m,n;q}(T_1,T_2)=&\ T_1 q^{m+n}(1 - T_1^{m+n-1} T_2 q^{mn+1}q^{(k-1)(m+n)}) \\&\cdot(1 - T_1^n T_2 q^{(m-1)(n+1)+1}q^{(k-1)(m+n)}) \\& \cdot (1 - T_1^m T_2 q^{(n-1)(m+1)+1}q^{(k-1)(m+n)}) (1 - T_2 q^{mn}).
    \end{align*}
    
\end{theorem}

This provides a direct formula for the bivariate conjugacy class zeta function of the $k$-fold central product of groups $G_{m,n}$. In particular it is not a direct scaling of the function for $G_{m,n}$, unlike the complementary representation zeta function, cf. Remark \ref{endremark}.
\newpage
\subsection{Notation} The following table summarises some frequently used notations.
\renewcommand{\arraystretch}{1.2}
\begin{table}[h]
    \centering
    
    \begin{tabular}{c|l|c}
        $\mathbb{N}$ & $\{1,2,\dots\}$\\
        $\mathbb{N}_0$ & $\{0,1,2,\dots\}$\\
        $[n]$& $\{1,2,\dots,n\}$\\ \hline
        $K$ & number field \\
        $\mathcal{O}$ & ring of integers of $K$ \\
        $\mathfrak{p}$ & non-zero prime ideal of $\mathcal{O}$ \\
        $\mathfrak o =\mathcal{O}_\mathfrak{p}$ & completion of $\mathcal{O}$ at $\mathfrak{p}$ \\
        $q$ & cardinality of $\mathcal O/\mathfrak p$\\
        \hline
        $\zeta_G^{\mathrm{cc}}(s)$&Conjugacy class zeta function&\ref{zetaorigdef}\\
        $\zeta_G^{\mathrm{irr}}(s)$&Representation zeta function&\ref{zetaorigdef}\\
        $\zeta_G^{\widetilde{\mathrm{irr}}}(s)$&Twist representation zeta function&\ref{twistzeta}\\
        $\zeta_G^{\mathrm{k}}(s)$&Class number zeta function&\ref{classnumber}\\
        \hline
        $\mathcal G_{m,n}$ & $\mathbb{Z}$-Lie lattice&\ref{Gmn}\\
        $G_{m,n}$ & Associated groups to $\mathcal G_{m,n}$&\ref{malcev}\\
        $\times_Z^kG$ & $k$-fold central product of $G$&\ref{k-fold}\\
        $\mathcal A(\boldsymbol{X})$ & $A$-commutator matrix&\ref{comms}\\
        $\mathcal B(\boldsymbol{X})$ & $B$-commutator matrix&\ref{comms}\\ \hline
        $M^\ast$ & $M\smallsetminus\mathfrak pM$&\ref{zirr}\\
        $W_d(\mathfrak o)$& $(\mathfrak o^d)^\ast$&\ref{zirr}\\
        $W_{d,N}(\mathfrak o)$&$((\mathfrak o/\mathfrak p^N)^d)^\ast$&\ref{zirr}\\ \hline
        $\nu_\mathfrak{p}$ & $\mathfrak{p}$-adic valuation \\
        $|\cdot|_\mathfrak{p}$ & $\mathfrak{p}$-adic norm \\
        $||\{x_i\}_{i \in I}||_\mathfrak{p}$ & max$_{i \in I}(|x_i|_\mathfrak{p})$ \\
    \end{tabular}
    \label{tab:my_label}
\end{table}
\renewcommand{\arraystretch}{1}

\section{Preliminaries}\label{sec2} %drafted


In this section we introduce machinery used to study representation zeta functions. We define central products and Lie lattices, and then provide a method for constructing $\mathbb Z$- Lie lattices from $\mathcal T$-groups and the converse. We outline the construction of a suitable basis for commutator matrices, which will then be used in the calculation of the bivariate zeta functions in Sections \ref{zirr} and \ref{zcc}. 


\subsection{Central Product}
\begin{define}
    Let $G$ and $H$ be groups and $\phi : Z(G) \to Z(H)$ an isomorphism between their centres. The \textit{canonical central product} is defined to be
    $$G \times_Z H \coloneq G \times H\big/\{(z, \phi(z^{-1})) : z \in Z(G) \}.$$
\end{define}

Let $G$ and $H$ be groups with nilpotency class at most 2, isomorphic centres, and presentations
$$G = \left\langle g_1, \dots ,g_n,z_1,\dots,z_d \:|\:G' \right\rangle, \: H = \left\langle h_1, \dots ,h_m,z_1,\dots,z_d \:|\:H' \right\rangle,$$
%
where $G'$, $H'$ denote the set of non zero commutators, and $z_1,\dots,z_d$ generate their centres. Their central product has presentation 
$$G \times_Z H = \left\langle g_1, \dots ,g_n,h_1,\dots,h_m,z_1,\dots,z_d \:|\:G',H' \right\rangle,$$
where $g_i \in G \times_Z H$ represents $g_i \in G$ under the embedding $$G \xhookrightarrow{} G \times_ZH, g \mapsto (g,1).$$
Note that $(z_i, 1)$ and $ (1,z_i)$ are equivalent under the central product quotient.
\begin{define}\label{k-fold}
    Let $k\in \mathbb N$. A \textit{k-fold canonical central product} for a group $G$ with centre $Z$, is defined to be
    \[ \times_Z^k \: G \coloneq \underbrace{G \times_Z G \times_Z \dots \times_Z G}_{k\ \mathrm{times}},\]
    where $$ G \times_Z G \coloneq  G \times G\big/(\{(z,z^{-1}) : z \in Z\}).$$
\end{define}


\subsection{Lie lattices}
A $\mathcal{T}$-group is a finitely generated, torsion-free, and nilpotent group. Throughout, we will be considering $\mathcal{T}_2$-groups, which are $\mathcal T $-groups of nilpotency class 2. Being finitely generated and nilpotent means that $\mathcal T$-groups are a type of polycyclic group, i.e.~they admit a subnormal series
$$1 = G_0 \triangleleft G_1 \triangleleft \dots \triangleleft G_n=G,$$
where the factors $G_{i+1}\big/G_i$ are cyclic. We define the \textit{Hirsch length} $h(G)$ of $G$ as the number of infinite factors in its subnormal series.

\begin{example}\textit{Heisenberg Group}\\
The (discrete) Heisenberg group $H$ consists of the unit upper triangular matrices and has the presentation 
$$ \left\langle x, y, z \: |\: [x,y] = z \right\rangle,$$
where any brackets not shown are equal to the identity. Clearly this is a nilpotent group of class 2. For copies $H_i = \left\langle x_i,y_i,z_i \:|\: [x_i,y_i]=z_i \right\rangle $, of the Heisenberg group, the $k$-fold canonical central product has presentation
$$ \times_Z^kH =\left\langle \{x_i, y_i\}_{i \in [1,k]}, z \: |\: \{[x_i,y_i] = z\}_{i \in [1,k]} \right\rangle.$$
We can calculate the Hirsch length via the lower central series
$$G \geq G^1 \geq G^2 \geq \dots \geq G^n =1,$$
where $G^1=[G,G], \ G^{i+1} = [G^i, G]$.
Then $h(G) = \sum^{n-1}_{i=1}\mathrm{rank}\big(G^i\big/G^{i+1}\big)$.
The Heisenberg group has a lower central series
$$H\geq Z(H) \geq 1.$$
Since $H\big/Z(H) \cong \mathbb Z^2$ and $Z(H) \cong \mathbb Z$, we have that $h(H)=3$.
\end{example}

Let $\mathcal O$ be a ring of integers over a number field $K$. An \textit{$\mathcal O$-Lie lattice} is a free and finitely generated $\mathcal O$-module equipped with a Lie bracket. Let $\Lambda$ be a nilpotent $\mathcal O$-Lie lattice of nilpotency class $c$. In \cite[Section 2.1.2]{zeta1}, Stasinski and Voll construct a unipotent group scheme $\mathbf G_{\Lambda}$ when $\Lambda' \subseteq c! \Lambda$ via a Hausdorff series so that $\mathbf G_{\Lambda}(\mathcal O)$ is a $\mathcal T$-group of nilpotency class c. They also explicitly construct a unipotent group scheme in the case $c=2$ which coincides with $\mathbf G_{\Lambda}$ when $\Lambda' \subseteq c! \Lambda$. We follow a version of this construction (cf.~\cite{snocken}) via a Mal'cev basis where $\mathcal O$ is taken as $\mathbb Z$. This is given by the correspondence proved by Mal'cev in \cite{malcev}. This is sufficient for the purposes of this paper, since we are considering families of $\mathbb Z$-Lie lattices, but for a general construction over $\mathcal O$ see \cite[Section 2.1.2]{zeta1}.

\subsubsection{Mal'cev correspondence}\label{malcev} First we will construct a $\mathbb Z$-Lie lattice from a $\mathcal T_2$-group G. Define $m=h(G\big/Z(G))$ and $n=h(Z(G))$. A \textit{Mal'cev basis} is a basis $x_1,\dots,x_m,x_{m+1},\dots,x_{m+n}$ such that $\overline{x_1},\dots,\overline{x_m}$ is a basis for $G\big/Z(G)$ (where $\overline{x_i}$ denotes the image of $x_i$ under the natural quotient map), and $x_{m+1}, \dots, x_{m+n}$ is a basis for $Z(G)$. Then, $G$ has a presentation
$$\Bigg<\begin{array}{c|}x_1,\dots,x_m\\x_{m+1},\dots,x_{m+n}
\end{array}\phantom{X} [x_i,x_j]=\prod^n_{k=1}x_{m+k}\lambda_{ij}^k\Bigg>,$$
where $\lambda_{ij}^k$ depend on our choice of Mal'cev basis. Note that since $G$ has nilpotency class 2, $G\big/Z(G)$ is abelian and $G\big/Z(G) \oplus Z(G)\cong \mathbb Z ^{m+n}$. This, as a $\mathbb Z$-module, equipped with the commutator bracket extended over the direct sum by anti-symmetry and bi-linearity, forms a $\mathbb Z$-Lie lattice. We will denote it by $\Lambda_G$.

Now we will construct a group $G_\Lambda$ from a $\mathbb Z$-Lie lattice $\Lambda$. Elements of $\Lambda$ are of the form $\mathbf x^\mathbf{a}=a_1x_1 + \dots+a_{m+n}x_{m+n}$ for $a_i \in \mathbb Z$, with $\mathbf x^\mathbf{a}+\mathbf x^\mathbf{a'}=\mathbf x^\mathbf{a+a'}$. We define a group multiplication $\star$ on the basis of $\Lambda$ as follows
%
$$x_i^{a_i} \star x_j^{a_j}= x_i^{a_i}x_j^{a_j}\prod^n_{k=1}x_{m+k}^{a_ia_j\lambda^k_{ij}}.$$
%
Then $(\Lambda,\star)\cong G_\Lambda$. These constructions are inverse to each other.

\subsection{Explicit basis construction}\label{basisconstruct}
Throughout, let $\Lambda$ be a nilpotent $\mathcal O$-Lie lattice, $\mathfrak p$ be a non-zero prime ideal in $\mathcal O$, and let $\mathfrak o =\mathcal O_\mathfrak p$ be the localisation at $\mathfrak p$. Consider the Lie lattice $\mathfrak g \coloneq \Lambda \otimes_\mathcal O \mathfrak o$ with derived lattice $\mathfrak g'$ and centre $\mathfrak z$.
\begin{define}
Let $M$ be a finitely generated $R$-module with submodule $N$. We call $\iota(N)$ the \textit{isolator} of $N$ in $M$, and define it to be the smallest submodule of $M$ such that $N \leq \iota(N) \leq M$ and $M\big/\iota(N)$ is torsion free.
\end{define} 
For a DVR $\mathfrak o$ with maximal ideal $(\pi)$, the elementary divisor theorem (see, for example, \cite[Theorem 6.12]{hungerford1974algebra}) allows us to choose a basis $\textbf{e}=\{e_1,\dots,e_n\}$ of $\mathfrak g $ such that, for any submodule $\mathfrak w $ and its isolator $\iota(\mathfrak w)$, we can express their bases in the form
\begin{align*}
\mathfrak w = \langle\pi^{b_i}e_i, \dots \pi^{b_j}e_j\rangle,&&\iota(\mathfrak w) = \langle e_i, \dots e_j\rangle,
\end{align*}
%
where $b_i, \dots,b_j$ are the \textit{elementary divisors} of $\mathfrak w$. As in \cite[Section 2.2.2]{zeta2}, we use this method to get a suitable basis for $\mathfrak g$. For context, we will outline the method for general nilpotency class. Let
\begin{align*}
    h &= \mathrm{rk}_{\mathfrak{o}} \mathfrak{g}, & 
    k &= \mathrlap{\mathrm{rk}_\mathfrak{o}(\iota(\mathfrak{g}') / \iota(\mathfrak{g}' \cap \mathfrak{z})) = \mathrm{rk}_\mathfrak{o}(\iota(\mathfrak{g}' + \mathfrak{z}) / \mathfrak{z}),}\\
    d &= \mathrm{rk}_\mathfrak{o}(\mathfrak{g}'), & 
    r - k &= \mathrm{rk}_\mathfrak{o}(\mathfrak{g} / \iota(\mathfrak{g}' + \mathfrak{z})), & 
    r &= \mathrm{rk}_\mathfrak{o}(\mathfrak{g} / \mathfrak{z}).
\end{align*}
%
Let $\overline{x}$ denote the image of $x$ under natural surjection \( \mathfrak{g} \to \mathfrak{g}/\mathfrak{z} \). As above, let $\pi$ be a uniformiser of $\mathfrak o$, and note that $\iota(\mathfrak{z})=\mathfrak{z}$ by \cite[Lemma 2.5]{zeta2}. We choose a basis $\textbf{e}$ for $\mathfrak g$, and $b_1,\dots,b_d\in\mathbb N$ such that
% $$
% \textbf{e} = (e_1, \ldots, e_{r-k}, \underbrace{e_{r-k+1}, \ldots, e_r}_{\iota(\mathfrak{g}' + \mathfrak{z})}, \overbrace{\underbrace{e_{r+1}, \ldots, e_{r-k+d}}_{\iota(\mathfrak{g}' \cap \mathfrak{z})}, e_{r-k+d+1}, \ldots, e_{h}}^{\mathfrak z})
% $$
\begin{align*}
&\textbf{e}=(e_1,\dots,e_h),&&\mathfrak{z} = \langle e_{r+1}, \ldots, e_{h} \rangle,\\
&\overline{\mathfrak{g}' + \mathfrak{z}} = \langle \pi^{b_1}e_{r-k+1}, \ldots, \pi^{b_k}e_r \rangle ,&& \iota(\mathfrak{g}' + \mathfrak{z}) = \langle \overline{e_{r-k+1}}, \ldots, \overline{e_r} \rangle,\\
&\mathfrak{g}' \cap \mathfrak{z} = \langle \pi^{b_{k+1}}e_{r+1}, \ldots, \pi^{b_d}e_{r-k+d} \rangle, && \iota(\mathfrak{g}' \cap \mathfrak{z}) = \langle e_{r+1}, \ldots, e_{r-k+d} \rangle.
\end{align*}
%
From this we can define a basis $\textbf{f} $ of $\mathfrak g'$ 
\begin{align*}
    &(\overline{f_1}, \ldots, \overline{f_k}) = (\pi^{b_1}e_{r-k+1}, \ldots, \overline{\pi^{b_k}e_r}),\\
    &(f_{k+1}, \ldots, f_d) = (\pi^{b_{k+1}}e_{r+1}, \ldots, \pi^{b_d}e_{r-k+d}).
\end{align*}
%
In the case that $\mathfrak{g}$ is a class-2 nilpotent  $\mathfrak o$-Lie lattice, the above reduces to
\begin{align*}
    &h= \mathrm{rk}_{\mathfrak o}\mathfrak g,
    &&k=\mathrm{rk}_\mathfrak{o}(\iota(\mathfrak g')\big/\iota(\mathfrak g' \cap \mathfrak z)) = \mathrm{rk}_\mathfrak{o}(\iota(\mathfrak g'+\mathfrak z)\big/\mathfrak z)= 0,
    \\&d = \mathrm{rk}_\mathfrak{o}(\mathfrak g') = \mathrm{rk}_\mathfrak{o}(\mathfrak z),
    &&r= \mathrm{rk}_\mathfrak{o}(\mathfrak g\big/\mathfrak z) =\mathrm{rk}_\mathfrak{o}(\mathfrak g\big/\iota(\mathfrak g' + \mathfrak z)),
\end{align*}
$$\text{with $\mathfrak o$-basis }\textbf{e} =\{\underbrace{e_1,\dots,e_r}_{\mathfrak g/\mathfrak z}, \underbrace{e_{r+1},\dots,e_h}_{\mathfrak z} \}. $$
Since $\iota(\mathfrak g')=\iota(\mathfrak z) = \mathfrak g'$, the basis $\textbf{f}=\{e_{r+1},\dots,e_h\}$ is a basis for $\mathfrak g'$. We define the structure constants $\lambda_{ij}^l$ with respect to this basis
$$[e_i,e_j]=\sum^d_{l=1}\lambda^l_{ij}f_l.$$ 
This allows us to define the following.

 
\subsection{Commutator matrices}
\begin{define}[{\cite[Definition~2.1]{commmatrix}}]\label{comms}
    We define the following \textit{commutator matrices} of $\mathfrak{o}$-linear forms,
    \begin{align*}
        \mathcal A_{\Lambda,\mathbf{e}}(\boldsymbol{X}) =\bigg(\sum^r_{l=1}\lambda^j_{il}X_l\bigg)_{ij} 
        \in \mathrm{Mat}_{r\times d}(\mathfrak o[\boldsymbol X]), \\
        \mathcal B_{\Lambda,\mathbf{e}}(\boldsymbol{X)} =\bigg(\sum^d_{l=1}\lambda^l_{ij}X_l\bigg)_{ij}
        \in \mathrm{Mat}_{r\times r}(\mathfrak o[\boldsymbol X]),
    \end{align*}
    with respect to a basis $\mathbf{e}$ and structure constants from the associated Lie lattice $\Lambda\otimes_{\mathcal O}\mathfrak o$. Where it is clear from context we may drop the basis $\mathbf{e}$, and refer to the commutator matrix of a lattice $\Lambda$ via its group (obtained by $\mathbf G_{\Lambda}(\mathcal O)$).
\end{define}

Recall our example of the Heisenberg group, $H$. Let $\mathfrak h$ denote its associated $\mathbb Z$-Lie lattice, with centre $\mathfrak z_\mathfrak h$. We can find the commutator matrix of its associated Lie lattice with respect to the basis $\textbf{e} = \{x,y,z\}$,


$$\begin{rcases} d = rk(\mathfrak h')=1 \\ r =rk(\mathfrak h/\mathfrak{z}_\mathfrak h)=2 \\ [x,y] = z\\ [y,x] = -z\end{rcases} \implies \mathcal{B}_{H,\textbf{e}}(\text{Y}) = \begin{pmatrix}
0 & Y\\
-Y & 0
\end{pmatrix}.$$


We can also find the commutator matrix of $\times_Z^kH$ with respect to the basis $\mathbf{f} = \{x_1,y_1,\dots,x_k,y_k,z\}$, noting that elements outside the centre from different copies have commutator equal to zero. This is given by

$$\begin{rcases} d = rk((\times_Z^kH)')=1 \\ r =rk(\times_Z^kH/\mathfrak{z})=2k \\ [x_i,y_i] = z\\ [y_i,x_i] = -z \end{rcases} \implies\mathcal{B}_{\times_Z^kH,\textbf{f}}(\textbf{Y}) =
\begin{pmatrix}
\mathcal{B}_{H,\textbf{e}}(\text{Y}_1) & &  \\
& \ddots & \\
 & & \mathcal{B}_{H,\textbf{e}}(\text{Y}_k)
\end{pmatrix}.$$
%
By reordering $\mathbf{f}$ as $\mathbf{f}' = \{x_1,x_2,\dots,y_{k-1},y_k,z\}$, we can find an alternate form, which is equivalent under elementary row and column operations,
%
$$\mathcal{B}_{\times_Z^kH,\textbf{f}'}(\textbf{Y}) =
\begin{pmatrix}
0&\textbf{Y}\cdot \textbf{I}_k \\
-\textbf{Y}\cdot \textbf{I}_k &0 
 \end{pmatrix}.
$$
This result can be naturally extended to $\mathcal T_2$-groups in general.
\begin{prop}\label{kfoldBmatrix}
    For any $\mathcal T_2$-group G with central product $\times^k_ZG$, its commutator matrix may be given as follows
    $$\mathcal B_{\times_Z^kG,\textbf{e}'}(\boldsymbol{Y}) = \bigoplus_{i=1}^{k}\mathcal B_{G, \textbf{e}}(\boldsymbol{Y}) =
    \begin{pmatrix}
\mathcal B_{G,\textbf{e}}(\boldsymbol{Y}) & &  \\
& \ddots & \\
 & & \mathcal B_{G,\textbf{e}}(\boldsymbol{Y})
\end{pmatrix}.$$
\end{prop}
\begin{proof}
    Given a basis $\mathbf{e} = \{g_1,\dots,g_r,z_1,\dots,z_d\}$ for $G$, we have a presentation
    $$G = \langle g_1,\dots,g_r,z_1,\dots,z_d| G' \rangle.$$
    Its $k$-fold central product has presentation
%
    $$\times_Z^kG = \left\langle \{g_{i1},\dots g_{ir}\}_{i\in[k]},z_1,\dots,z_d | \{G_i'\}_{i\in(1,\dots,k)}\right\rangle.$$
    Where $\mathbf{e}' = \{\{g_{i1},\dots,g_{ir}\}_{i\in[k]},z_1,\dots,z_d\}$ is our basis ordering for \\$\mathcal B_{\times_Z^kG,\textbf{e}'}(\boldsymbol{Y})$, and each $g_{ij}$ representing $g_j$ under the embedding
    %
    $$G \xhookrightarrow{} \times_Z^kG, \; g_j \mapsto (1,\dots,g_{ij},\dots,1),\; i\in[k].$$

    Note $\{\dots,z_{il},\dots\}$ and $\{\dots,z_{jl},\dots\}$ are equivalent under the central product quotient and are therefore not distinct in the presentation. Elements outside the centre from different copies do not interact i.e.~they are not identified in the presentation hence their commutator is zero. Therefore our basis $\mathbf{e}$ induces a block diagonal matrix, with blocks determined by the commutator matrix of $G$.
\end{proof}


\section{Computing the representation zeta function}\label{zirr}


In \cite{zeta2}, a link was established between representation zeta functions of groups and Poincaré series encoding properties of their associated Lie lattices. This can then in turn be expressed as a $\mathfrak p$-adic integral, which in some cases allows direct calculation. For the purposes of proving \eqref{kfoldirr}, we only need the Poincaré series expression, however in Section~\ref{zcc} we will directly compute the integral resulting from the conjugacy class zeta function (cf. Theorem \ref{BigTheorem}). 

\subsection{$\mathcal B(\mathbf X)$ and its Poincaré series} For groups of the form $\mathbf G(\mathcal O_\mathfrak p)$, and given conditions on $p$ (the cardinality of the residue field), we can apply the \textit{Kirillov orbit method}. This was developed for $\mathcal T$-groups by Howe~\cite{Howe1977ONRO}, and allows us to construct the irreducible representations via co-adjoint orbits, which can then be expressed as a Poincaré series. For $\mathcal T_2$-groups, this is applicable for all $p$ \cite[Section 2.4]{zeta1}.

Recall that, given a non-zero prime ideal $\mathfrak p$, $\mathfrak o = \mathcal O_\mathfrak p$, and $q=|\mathcal O/\mathfrak p|$. For an $\mathfrak o$-module $M$ let us write $M^\ast \coloneq M \smallsetminus \mathfrak p M$ (when $M$ is trivial we set $\{0\}^\ast = \{0\}$). Then given $d, N \in \mathbb N$, define 
\begin{align*}
&W_d(\mathfrak o) = (\mathfrak o^d)^\ast&\text{and}&&W_{d,N}(\mathfrak o)= ((\mathfrak o/\mathfrak p^N)^d)^\ast.
\end{align*}
Recall the $\mathcal B$-commutator matrix, from Definition~\ref{comms},
$$\mathcal B(\mathbf{X)} =\bigg(\sum^d_{l=1}\lambda^l_{ij}X_l\bigg)_{ij}
        \in \mathrm{Mat}_{r\times r}(\mathfrak o[\mathbf X]).$$
%
We define a sub matrix consisting of the last $t$ columns of $\mathcal B(\mathbf{X)}$,
%
$$\mathcal B_t(\mathbf{X)}=\big(\mathcal B(\mathbf{X)}_{ij}\big)_{i\in[r],j\in[r-k+1,r]}.$$
%
We say a matrix $M \in \mathrm{Mat}_{r\times t}$ has \textit{elementary divisor} type $\mathbf c = (c_1, \dots, c_t)$ if $M$ has the Smith normal form
$$\begin{pmatrix}
    \pi^{c_1}&&\\
    &\ddots&\\
    &&\pi^{c_t}\\&&
\end{pmatrix},$$
i.e.~it is equivalent via elementary row and column operations to the above matrix, where $\pi^{c_1}|\dots|\pi^{c_t}$ and therefore $0\leq c_1\leq\dots\leq c_t$. We denote this $\nu(M) = \mathbf c$. Note that since $\mathcal B(\mathbf{X)}$ is an anti-symmetric matrix, its elementary divisors must come in pairs, with a zero row and column if $r$ is odd. Let $\mathbf{y} \in W_d(\mathfrak o)$. Then the elementary divisors of $\mathcal B(\mathbf{y)}$ are of the form $\mathfrak p^a$ for $a \in \mathbb N \cup \{\infty\}$ (where $\mathfrak p^\infty$ would be an elementary divisor equal to zero), and we have
%
\begin{align*}
    &\nu(\mathcal B(\mathbf y))\coloneq \mathbf a = (a_1,\dots,a_{\lfloor r/2\rfloor}),
    \\&\tilde\nu(\mathcal B(\mathbf y))\coloneq \mathbf a=\begin{cases}
    (a_1, a_1, \dots,a_{r/2},a_{r/2}) & \text{if $r$ even}\\
    (a_1, a_1, \dots,a_{(r-1)/2},a_{(r-1)/2},\infty) & \text{if $r$ odd}.
\end{cases}
\end{align*}
%\textcolor{red}{prerequisite on elementary divisor/within dvr?d needs to depend on rk(G') given in $\mathcal{N}$}
%Define $W(\mathfrak{o}) \coloneq (\mathfrak{o}^d)^\ast \coloneq (\mathfrak{o}^d \smallsetminus \mathfrak{po}^d)$. For $\mathbf{y} \in W(\mathfrak{o})$, the matrix $\mathcal B(\mathbf{y})$ has elementary divisors of the form $\mathfrak{p}^a$ for $a \in \mathbb{N}_0 \cup \{\infty\}$. pairs/inf
%
Let $\overline{\mathbf{y}}$ be the image of $\mathbf{y}$ under the mapping $$(\mathfrak{o}^d)^\ast\longrightarrow ((\mathfrak{o}/\mathfrak{p}^N)^d)^\ast = W_{d,N}(\mathfrak{o}).$$ Then $\mathcal B(\overline{\mathbf y})$ is an antisymmetric matrix over $\mathfrak o \big/\mathfrak p^n$ and we get the elementary divisor type
%
$$\nu(\mathcal B(\overline{\mathbf{y}})) =( min\{a_i, n\}_{i \in \{1,\dots,\left\lfloor d/2 \right\rfloor\}}).
$$
%
We define the counting function for the size of subsets of $W_N(\mathfrak o)$, partitioned by divisor type,
%
$$\mathcal{N}^\mathfrak{o}_{N, \mathbf{a},\mathbf c} \coloneq \# \{ \mathbf{y} \in W_{d,N}(\mathfrak{o}) \mid \nu(\mathcal B(\mathbf y)) = \mathbf{a}, \tilde\nu(\mathcal B_t(\mathbf y)\cdot\mathrm{diag}(\pi^{b_1},\dots,\pi^{b_t})) = \mathbf c \}.$$
%
This allows us to define the following Poincaré series
$$\mathcal P_{\mathcal B,t, \mathfrak o}(s) \coloneq \sum_{\substack{N \in \mathbb{N}_0,\\ \mathbf{a} \in \mathbb{N}_0^{\lfloor r/2 \rfloor},\, \mathbf{c} \in \mathbb{N}_0^t}} 
\mathcal{N}^{\mathfrak o}_{N, \mathbf{a}, \mathbf{c}} q^{-\sum_{i=1}^{\lfloor r/2 \rfloor}(N - a_i)s - \sum_{i=1}^t (N - c_i)}.$$

The link between this series and the representation zeta function was first proved in \cite{zeta2}, but the analogue for our specific unipotent group schemes was given in \cite{zeta1}. The case for general nilpotency class ($>3$) is given as follows.

\begin{prop}[{\cite[Proposition 2.9]{zeta1}}] If $p$ is odd or $p=2$, and $c>3$, then 
    $$\zeta^{\widetilde{\mathrm{irr}}}_{\mathbf G(\mathfrak o)}(s)=\mathcal P_{\mathcal B,t,\mathfrak o}(s).$$
\end{prop}
%
\subsubsection{Nilpotency class 2} In nilpotency class 2 we only need a simpler counting function, which is satisfied without the $p=2$ condition,
%
$$\mathcal{N}^\mathfrak{o}_{N, \mathbf{a}} \coloneq \# \{ \mathbf{y} \in W_{d,N}(\mathfrak{o}) \mid \nu(\mathcal B(\mathbf{y})) = \mathbf{a} \}.$$
%
We then get the following correspondence.
%
\begin{prop}[{\cite[Proposition 2.18]{zeta1}}]\label{p1} Let $\mathbf G(\mathfrak o)$ be of nilpotency class 2. Then, for all primes $p$,
    $$
\zeta^{\widetilde{\mathrm{irr}}}_{\mathbf G(\mathfrak{o})}(s) = 
\sum_{N \in \mathbb{N}_0, \, \mathbf{a} \in \mathbb{N}_0^{ 
\left\lfloor \frac{r}{2} \right\rfloor}}
\mathcal{N}_{N,a}^\mathfrak{o} \: q^{
-\sum_{i=1}^{\lfloor r/2 \rfloor} 
(N - a_i)s }
\eqcolon \mathcal{P}_{\mathcal B, \mathfrak{o}}(s).$$
\end{prop} 

\subsection{Application to $k$-fold central product} We have now in a position to recover the following result, giving the precise relation between the local factors of $G$ and those of $\times_Z^kG$.
\begin{prop}\label{prop} Let $G$ be a $\mathcal T_2$-group, and denote by $\times^k_ZG$ its $k$-fold central product. Then
%
    $$\zeta^{\widetilde{\mathrm{irr}}}_{\times_Z^kG}(s) = \zeta^{\widetilde{\mathrm{irr}}}_{G}(ks).$$
\end{prop}
\begin{proof}
    Via the construction in \ref{malcev}, we have $G= \mathbf G(\mathcal O)$ for some unipotent group scheme $\mathbf G$, and ring of integers $\mathcal O$. For readability we will denote $\times^k_ZG= \times_Z^k\mathbf G(\mathcal O)$ by $H=\mathbf H(\mathcal O)$. This correspondence is necessary to refer to their local factors. 
    
    Clearly, since $\mathbf G(\mathcal O)$ has nilpotency class 2, if $\mathcal B_{G}(\mathbf{y})$ has elementary divisor type $\mathbf{a} = \{a_1,\dots,a_n\}$, then $\mathcal B_{H}(\mathbf{y}) = \bigoplus_{i=1}^{k}\mathcal B_{G}(\mathbf{y})$ , will have elementary divisor type $$\mathbf{a}^k = \{\underbrace{a_1,\dots,a_n, a_1,\dots,a_n\}}_{k\text{ copies}}.$$ Then,

\begin{align*}   
\nu(\mathcal B_G(\mathbf{Y})) = \mathbf{a} &\iff \nu(\mathcal B_{H}(\mathbf{Y})) = \mathbf{a}^k\\
&\implies \mathcal{N}_{N,\mathbf{a},G}=\mathcal{N}_{N, \mathbf{a}^k, H}\\
\implies \zeta^{\widetilde{\mathrm{irr}}}_{\mathbf H(\mathfrak o)}(s) &\underset{\ref{p1}}{=} 
\sum_{N \in \mathbb{N}_0, \, \mathbf{a} \in \mathbb{N}_0^{kn}}\mathcal{N}_{N,\mathbf{a},H} \: q^{-\sum_{i=1}^{kn} (N - a_i)s}\\ &= 
\sum_{N \in \mathbb{N}_0, \, \mathbf{a} \in \mathbb{N}_0^{n}}\mathcal{N}_{N,\mathbf{a}^k,H} \: q^{-k\sum_{i=1}^{n} (N - a_i)s}\\ &=
\sum_{N \in \mathbb{N}_0, \, \mathbf{a} \in \mathbb{N}_0^{n}}\mathcal{N}_{N,\mathbf{a},G} \: q^{-k\sum_{i=1}^{n} (N - a_i)s} \underset{\ref{p1}}{=} 
\zeta^{\widetilde{\mathrm{irr}}}_{\mathbf G(\mathfrak o)}(ks).
\end{align*}
This holds for all local factors, hence by the Euler product \eqref{euler},

$$\zeta^{\widetilde{\mathrm{irr}}}_{\times_{Z}^{k}\mathbf G(\mathcal{O})}(s) = \zeta^{\widetilde{\mathrm{irr}}}_{\mathbf G(\mathcal{O})}(ks).$$
\end{proof}
%
This result establishes a strong link between the twist-equivalent representation of $\mathcal T$-groups and their central products. We will now investigate how this relation extends to both the representation and conjugacy class bivariate zeta functions, however it would be interesting to see whether Proposition \ref{prop} or an analogous statement holds in the more complex case of nilpotency class greater than 2.

% Since $|\mathcal{O}:I|$ is independent of our choice of group scheme, directly applying Proposition \ref{prop}  gives
% \begin{coroll}\label{ZIRR}
%     $$\mathcal Z^{\mathrm{irr}}_{\times^k_ZG_{m,n}}(s_1,s_2) = \mathcal Z^{\mathrm{irr}}_{G_{m,n}}(ks_1,s_2)$$
% \end{coroll}
\section{Bivariate Zeta functions}\label{biv}



We begin this section with a brief corollary of Proposition \ref{prop}, before introducing \textit{Igusa-type integrals}. Building on from our expression in terms of Poincaré series, they provide a new way of computing these zeta functions. Our focus will be on the bivariate case, however this method exists and is well-established for the univariate zeta functions (cf. \cite[Section 2.2.3]{zeta1}, or \cite[Section 2.2]{voll2008}). Recall Definition \ref{bivfunc} of the bivariate zeta functions,
%
\begin{align*}\label{bivfunc2}
\mathcal{Z}_{\mathbf{G}(\mathcal{O})}^{\mathrm{irr}}(s_{1},s_{2})=\sum_{(0)\neq I\trianglelefteq\mathcal{O}}\zeta_{\mathbf{G}(\mathcal{O}/I)}^{\mathrm{irr}}(s_{1})|\mathcal{O}:I|^{-s_{2}},
\\
\mathcal{Z}_{\mathbf{G}(\mathcal{O})}^{\mathrm{cc}}(s_{1},s_{2})=\sum_{(0)\neq I\trianglelefteq\mathcal{O}}\zeta_{\mathbf{G}(\mathcal{O}/I)}^{\mathrm{cc}}(s_{1})|\mathcal{O}:I|^{-s_{2}}.
\end{align*}

\subsection{Extension to bivariate case} Proposition \ref{prop} can be extended to the bivariate representation zeta function via the specialization proved by Lins in \cite{LinsI}. Recall the relation for nilpotency class 2 (where $r$ is the rank of $\mathfrak g/\mathfrak z$),
$$\zeta^{\widetilde{\mathrm{irr}}}_{\mathbf G(\mathfrak o)}(s) = (1-q^{r-s_2})\mathcal Z^\mathrm{irr}_{\mathbf{G}(\mathfrak o)}(s_1,s_2)\bigg|_{\substack{s_1 \to s - 2 \\ s_2 \to r}}.$$
%
The following result is immediate.
\begin{coroll}\label{ZIRR} Let $\mathbf G(\mathfrak o)$ be of nilpotency class 2. Then,
$$(1-q^{r-s_2})\mathcal Z^\mathrm{irr}_{\mathbf G(\mathfrak o)}(s_1,s_2)\bigg|_{\substack{s_1 \to ks - 2 \\ s_2 \to r}}=(1-q^{r-s_2})\mathcal Z^\mathrm{irr}_{\times^k_Z\mathbf G(\mathfrak o)}(s_1,s_2)\bigg|_{\substack{s_1 \to s - 2 \\ s_2 \to r}}.$$
\end{coroll}
%
\begin{remark}
    We could extend Corollary \ref{ZIRR} to the global bivariate function $\mathcal Z^\mathrm{irr}_{\mathbf G(\mathcal O)}$ via the Euler product \eqref{biveuler} however, in this context, this provides no further insight.
\end{remark}


\subsection{Igusa-type integrals}\label{4.1}
In \cite[Section 4.1]{LinsI}, Lins shows (in an analogous way to Section 3) that both bivariate zeta functions can be expressed in terms of Poincaré series (cf. \cite[Proposition 4.7]{LinsI}). Voll demonstrates in \cite[Section 2.2]{voll2008} that we can reduce this problem to computing certain $p$-adic integrals associated with polynomials describing the degeneracy loci of our commutator matrices; specifically, these polynomials correspond to the minors of our commutator matrix. Lins extends this approach to the bivariate case (\cite[Section 4.2]{LinsI}), which we will briefly recap here.

Recall $W_d(\mathfrak o)=\mathfrak o^d\setminus \mathfrak p^d$, where $\mathfrak o$ is the completion of $\mathcal O$ at $\mathfrak p$. Define $\mathcal R(\bf X)$ as a matrix of linear forms in $\mathfrak o[\bf X]$, a generalised version of the commutator matrices $\mathcal A(\bf X)$ and $\mathcal B(\bf X)$.

For a given matrix, the \textit{$k$-minors} refer to the set of determinants for any selection of $k$ rows and $k$ columns. For example, given a matrix $\big(\alpha_{ij} \big) \in \mathrm{Mat}_{n\times m}(F)$, the $3$-minors would be the set of determinants
$$
 \begin{array}{|ccc|}
\alpha_{ap}&\alpha_{aq}&\alpha_{ar}\\
\alpha_{bp}&\alpha_{bq}&\alpha_{br}\\
\alpha_{cp}&\alpha_{cq}&\alpha_{cr}\\
\end{array} \phantom{XX}
\begin{array}{cc}
    \text{distinct } a,b,c \in [n]\\
     \text{distinct } p,q,r \in [m].
\end{array}
$$

Let $u_\mathcal R$ denote the rank of $\mathcal R(\bf X)$ over its field of fractions, and $\mathcal F^k(\mathcal R(\bf X))$ denote the ideal generated by $k$-minors of $\mathcal R(\bf X)$. Let $\mu$ denote the additive Haar measure (see, for example \cite[Section 3.1]{deitmar2013automorphic}), normalised so that $\mu(\mathfrak o^{d+1})=1$.

We define \customsmall
$$\mathcal{Z}_{\mathcal{R}}(\rho,\tau) = \frac{1}{1-q^{-1}}\int_{(w,{\bf x})\in p\times W_{d,N}(\mathfrak{o})}|w|_\mathfrak{p}^\tau\prod_{k=1}^{u_{\mathcal{R}}} \frac{\|\mathcal{F}^k(\mathcal{R}({\bf x}))\cup w\mathcal{F}^{k-1}(\mathcal{R}({\bf x}))\|_\mathfrak{p}^\rho}{\|\mathcal{F}^{k-1}(\mathcal{R}({\bf x}))\|_\mathfrak{p}^\rho}\, d\mu$$
\normalsize
so that we can express the bivariate zeta functions as follows.

\begin{prop}[{\cite[Proposition 4.8]{LinsII}}]\label{igusa} If either $c=2$ or $p>c>2$, then
\begin{align*}
\mathcal{Z}_{\mathbf{G}(\mathfrak o)}^{\mathrm{irr}}(s_1,s_2) &= \frac{1}{1-q^{r-s_2}}\Bigg(1+\mathcal{Z}_B\bigg(-\frac{s_1+2}{2},\frac{s_1+2}{2}\,u_B+s_2-h-1\bigg)\Bigg), \\
\mathcal{Z}_{\mathbf{G}(\mathfrak o)}^{\mathrm{cc}}(s_1,s_2) &= \frac{1}{1-q^{r-s_2}}\bigg(1+\mathcal{Z}_A\bigg(-(s_1+1),(s_1+1)u_A+s_2-h-1\bigg)\bigg).
\end{align*}
\end{prop}

We will use this to calculate the conjugacy class bivariate zeta function for $k$-fold central products of groups with associated $\mathbb Z$-Lie lattice from a generalisation of the family $\mathcal G_n$ (see Definition~\ref{famililes}). First we need to find the $\mathcal A$-commutator matrix.
    

\section{Computing the Bivariate conjugacy class function}\label{zcc}
\subsection{Computing the commutator matrix}
Recall the family of $\mathbb Z$-Lie lattices 
$$\mathcal{G}_{n} = \langle x_{k}, y_{ij} \mid [x_{i}, x_{n+j}] = y_{ij}, \ 1 \leq k \leq 2n, \ 1 \leq i, j \leq n \rangle.$$
%
We can generalise this to a larger family
%
$$\mathcal G_{m,n} \coloneq \left\langle x_1,\dots,x_{m+n}, \{z_{ij} \}_{i \in[m], j \in [n]} \:| \:[x_i,x_{m+j}] = z_{ij}\right\rangle.$$
%
Note that by setting $m=n$ we recover $\mathcal G_n$. Denote the group scheme $\mathbf{G}_{\mathcal G_{m,n}}(\mathcal O)$ by $G_{m,n}$. Its $k$-fold central product has presentation

\begin{equation}\label{hi}
\times_Z^k G_{m,n} = \stretchleftright[600]{\langle}
{\begin{array}{cc|c} x_{1\, 1}, \dots,x_{1\, m+n} &&[x_{1\, i},x_{1\, m+j}] = z_{ij} \\
\ \ \ \vdots \phantom{XXXXX} \vdots & \{z_{ij}\}_{i \in [m],j\in[n]}  & \phantom{XXXX}\vdots \\
x_{k\, 1}, \dots,x_{k\,m+n} &&[x_{k\, i},x_{k\, m+j}]=z_{ij}
\end{array}}{\rangle}.\end{equation}

Recall the commutator matrix from \ref{comms},
$$
\left( \mathcal A(\mathbf{X}) \right)_{ij} = \sum_{l=1}^{r}\lambda^j_{il}X_l \ \
\begin{array}{c}
    i \in [r]\\j\in[d]
\end{array} \ \in \mathrm{Mat}_{r\times d}(\mathfrak{o}[\mathbf{X}]).
$$

\begin{prop}[{\cite[Section 4.1]{z}}] The $\mathcal A$-commutator matrix, for groups of type $G_{m,n}$ has the following form
\customsmall
$$
\mathcal A_{G_{m,n}}(\mathbf{X})=\begin{pmatrix} 
X_{m+1} \dots X_{m+n} && \\
& \ddots & \\
&& X_{m+1} \dots X_{m+n} \\
-X_1\cdot \mathbf{I}_n &\dots& -X_m \cdot \mathbf{I}_n
\end{pmatrix} \ \in Mat_{m+n,mn}(\mathfrak{o}[\mathbf{X}]),
$$ 
\normalsize
where $\mathbf{I}_n$ denotes the $n\times n$ identity matrix and $\mathbf{X}$ is the $(m+n)$-tuple $X_1,\dots,X_{m+n}$. 

\emph{We recall the proof of this, for completeness.}
\end{prop}
\begin{proof}
Recall our basis for $G_{m,n}'$ is $z_{1\, 1}, \dots,z_{m\,n}$. We can relabel the indices so that it is ordered more clearly, $(i-1)n+j$, where $i \in [m], j \in [n]$. By substituting the correct values for $G_{m,n}$, we obtain 
%
$$\big(\mathcal A(\mathbf X)\big)_{ik}= \sum_{j=1}^{m+n}\lambda_{ij}^k\mathbf X_j\phantom{XX}\begin{array}{l}
i \in [m+n]\\k\in[mn]
\end{array}.$$
%
Consider the first $m$ rows. Clearly when $j\leq m$, $[x_i,x_j]$ is not in the presentation, hence $\lambda_{ij}^k=0$. We can therefore consider the terms $\lambda_{i,m+j}^k$ where $j\in[n]$. Since $[x_i, x_{m+j}]=z_{ij}$, these are non zero precisely when $k=(i-1)n + j$ (our index equivalent of $z_{ij}$) and they are equal to $1$ in this case. Hence
%
$$\big(\mathcal A(\mathbf X)\big)_{ik}= \begin{cases}
    X_{m+j}&\text{ when }k=(i-1)n+j\\0 &\text{ else}
\end{cases}.$$
%
Therefore, in row $i \in [m]$, we get sequential entries $X_{m+1}, \dots,X_{m+n}$ (as $j$ varies in $[n]$) and zeros elsewhere. This sequence begins in the column where $k$ corresponds to $j=1$, ie $k=(i-1)n+1$.

Now consider the last $n$ rows. We let $i\in[n]$ and consider rows $m+i$. By anti-symmetry, $[x_{m+i},x_j]= -[x_j,x_{m+i}]=-z_{ji}$. The middle term is non-zero precisely when $j\in[m]$ and $k=(j-1)n+i$, and is equal to $-1$ in this case. Hence
%
$$\big(\mathcal A(\mathbf X)\big)_{m+i,k}= \begin{cases}
    -X_{j}=-X_{\frac{k-i+n}{n}}&\text{ when }j\in[m], k=(j-1)n+i\\0 &\text{ else}
\end{cases}.$$
%
By iterating over $j$, it is clear to see that we get copies of $-X_j\mathbf I_n$ in columns $(i-1)n+1$ to $(i-1)n+j$. The required matrix follows.
\end{proof}

\subsection{$\mathcal A(\mathbf X)$ for central products}
Similarly to Proposition \ref{kfoldBmatrix}, we can easily extend this structure to $k$-fold central products.
\begin{prop} The $\mathcal A$-commutator matrix for groups $\times^k_ZG_{m,n}$ has the following form
    $$\mathcal A_{\times_Z^kG_{m,n}}(\overline{\mathbf{X}})=\begin{pmatrix}
\mathcal A_{G_{m,n}}(\mathbf{X_1}) \\ \vdots \\ \mathcal A_{G_{m,n}}(\mathbf{X_k})
\end{pmatrix}
\begin{rcases} \\ \\ \\ \\
\end{rcases}\text{k copies}
$$
where $\overline{\mathbf{X}}= (\underbrace{X_{1\,1},\dots,X_{1\,m+n}}_{\mathbf{X_1}},\dots,\underbrace{X_{k\,1},\dots,X_{k\,m+n}}_{\mathbf{X_k}}).$
\end{prop}
\begin{proof}
    By the definition of $\mathcal A(\mathbf{X})$, its columns correspond to the basis elements of the centre. The central elements are invariant under the central product quotient, and hence they are formed by the commutator of the same elements from  different copies of $G$ in $\times_Z^kG$. Note that this specific structure (row and column ordering) is with respect to the basis order in the presentation \eqref{hi}.
\end{proof}

For convenience, we will now consider the following reordered basis for $\times_Z^k G_{m,n}$,
\customsmall
$$\stretchleftright[600]{\langle}
{\begin{array}{ccc|c} x_{1\,1}, \dots,x_{1\, m} &x_{1\,m+1},\dots,x_{1\,m+n}&&[x_{1\,i},x_{1\,m+j}] = z_{ij} \\
\ \ \ \vdots \phantom{XXXXX} \vdots && \{z_{ij}\}_{i \in [m],j\in[n]}  & \phantom{XXXX}\vdots \\
x_{k\,1}, \dots,x_{k\,m} &x_{k\,m+1},\dots,x_{k\,m+n}&&[x_{k\,i},x_{k\,m+j}]=z_{ij}
\end{array}}{\rangle}.$$
\normalsize
With the respective commutator matrix \customsmall$\mathcal A_{\times_Z^kG_{m,n}}(\mathbf{X})\in Mat_{k(m+n),mn}(\mathfrak{o}[\mathbf{X}])$\normalsize,
\begin{align*}
\mathcal A_{\times_Z^kG_{m,n}}(\mathbf{X})=\begin{pmatrix} 
X_{1\,m+1} \dots X_{1\,m+n} \\
\vdots \hphantom{XXXXX} \vdots \\
X_{k\,m+1} \dots X_{k\,m+n}\\
& \ddots & \\
&& X_{1\,m+1} \dots X_{1\,m+n} \\
&&\vdots \hphantom{XXXXX} \vdots \\
&&X_{k\,m+1} \dots X_{k\,m+n}\\
-X_{1\,1}\cdot \mathbf{I}_n &\cdots& -X_{1\,m} \cdot \mathbf{I}_n\\
\vdots && \vdots \\
-X_{k\,1}\cdot \mathbf{I}_n &\cdots& -X_{k\,m} \cdot \mathbf{I}_n\\
\end{pmatrix}. 
\end{align*}

\subsection{Ideals of $t$-minors} This section will focus on computing the $t$-minors of $\mathcal A_{\times_Z^kG_{m,n}}(\mathbf{X})$, moreover showing that they can be generated from a set of monomials, allowing us to compute the integral in Section \ref{biv}.
\begin{define}
    We define $\mathcal{M}_{\times_Z^kG_{m,n}}^t$ as the set of monomials
    $$X_{i_1j_1}\dots X_{i_{\lambda}j_{\lambda}}X_{p_1\,m+q_1}\dots X_{p_{\omega}\,m+q_{\omega}},$$
    where $\lambda \in [n], \ \omega \in[m] $ such that  $\lambda +\omega = t$, $i_1,\dots,i_\lambda,p_1,\dots,p_\omega \in [k]$, $j_1,\dots,j_\lambda \in [m]$, and $q_1,\dots,q_\omega\in[n]$. Note that $i,j$ and $pm+q$ are not necessarily pairwise distinct.


\end{define}
Analogously to \cite[Lemma 4.8]{z}, we have the following result.
\begin{lemma}Let $f$ be the determinant of a $t$-minor of $\mathcal A_{\times^k_ZG_{m,n}}(\mathbf{X})$ formed by selecting $\omega$ rows from the first $k\cdot m$ rows and $\lambda$ rows from the remaining $k\cdot n$ rows of $\mathcal A_{\times^k_ZG_{m,n}}(\mathbf{X})$. Then $f$ is homogenous of degree $\omega$ with respect to $X_{1\,m+1}, \dots X_{k\,m+n}$ and homogenous of degree $\lambda$ with respect to $X_{1\,1},\dots,X_{k\,m}$.
\end{lemma}
\begin{proof} Let $a \in [m+1,m+n]$, $b\in [m]$, and $c \in [k]$, so that $X_{c\,a}$ is a generic element in $X_{1\,m+1},\dots,X_{k\,m+n}$, and $X_{c\,b}$ is a generic element in $X_{1\,1},\dots,X_{k\,m}$.
    From our basis reordering above, the first $k\cdot m$ rows of $\mathcal A_{\times^k_ZG_{m,n}}(\mathbf{X})$ have entries $X_{c\,a}$, and the remaining $k\cdot n$ rows have entries $X_{c\,b}$. Let $A= \big(a_{ij}\big)$ be the $t$-minor such that $\mathrm{det}(A)=f$. Then, the first $\omega$ rows have entries $X_{c\,a}$ and the other $\lambda$ rows have entries $X_{c\,b}$.
    Recall the Leibniz formula for the determinant
    $$\det(A) = \sum_{\sigma \in S_n} \text{sgn}(\sigma) \prod_{i=1}^n a_{i, \sigma(i)}.$$
    Then
    $$a_{i,\sigma(i)}= \begin{cases}
        X_{c\,a} & i \in [\omega]\\
        X_{c\,b} & i \in [\omega,\omega+\lambda].
    \end{cases}$$
    Hence the product of these $a_{i, \sigma(i)}$ (if non-zero) is homogenous of degree $\omega$ in $X_{1\,m+1},  \dots,\allowbreak X_{k\,m+n}$, and homogenous of degree $\lambda$ in $X_{1\,1},\dots,X_{k\,m}$. The sum of these (determinant of $A$) therefore also respects these conditions.
\end{proof}

In order compute the integral in Proposition~\ref{igusa}, we need the ideal generated by the $t$-minors of $\mathcal A_{\times_Z^kG_{m,n}}(\mathbf{X})$. 
We claim that a smaller set of monomials generates this ideal. This is a modified version of the proof in \cite[Proposition 4.9]{z}, given for $G_{m,n}$. We use the terminology \textit{$k$-sums} to refer to sums of up to $k$ elements from a given set. 
\begin{prop}
    The set of monomials $\mathcal{M}_{\times_Z^kG_{m,n}}^t$
    \begin{enumerate}
        \item generates (up to sign) all minors of $\mathcal A_{\times^k_ZG_{m,n}}$ as $k$-sums from \\$\mathcal{M}_{\times_Z^kG_{m,n}}^t$ \label{prop:i}
        \item contains only monomials of $\mathcal A_{\times^k_ZG_{m,n}}$ \label{prop:ii}
    \end{enumerate}

\end{prop}
\begin{proof}  For ease of notation, we denote $A_{m,n}(\mathbf{X}) \coloneq \mathcal A_{\times_Z^kG_{m,n}}(\mathbf{X})$. Following the proof in \cite[Proposition 4.9]{z} we proceed by induction on $m+n$ while keeping $k$ constant.
The base case is $m=n=1$, with the following commutator matrix,
$$A_{1,1}(\mathbf{X})=
\begin{pmatrix}X_{1\,2}\\\vdots\\X_{k\,2}\\-X_{1\,1}\\\vdots\\-X_{k\,1}\end{pmatrix}.$$
The only minors of this matrix are the individual terms, which are precisely the elements (up to sign) of $\mathcal{M}_{\times_Z^kG_{1,1}}^1$, hence the proposition holds. We split the induction into two cases.
\\
\textbf{Case 1} : $1 \neq m \geq n $\\
To prove \ref{prop:i}, let $M$ be a $t$-square submatrix of $A_{m,n}(\mathbf{X})$.
Since $2m > t$, there exists a  $ j \in[m] $ such that $X_{ij}$ is in at most one column of $M$ (this will be the same column for all $i\in[k]$). Define 
$$A_j(\mathbf{X})\coloneq\left(\begin{array}{c|c}
&0\\
&-X_{1\,j}\mathbf{I}_n\\
A_{m-1,n}(\mathbf{X}\smallsetminus \{X_{ij}\}_{i\in[k]})& -X_{2\,j}\mathbf{I}_n\\
&\vdots\\
&-X_{k\,j}\mathbf{I}_n\\
\hline
& X_{1\,m+1} \dots X_{1\,m+n} \\
0&\vdots \hphantom{XXXXX} \vdots \\
&X_{k\,m+1} \dots X_{k\,m+n}\\
\end{array}\right),$$
where, for a given $j$, $\mathbf{X}\smallsetminus \{X_{ij}\}_{i\in[k]}$ denotes the $(k\cdot(m+n-1))$-tuple obtained by removing $\{X_{1\,j},\dots,X_{k\,j}\}$ from $\mathbf{X}$. Then $M$ is a submatrix of $A_j(\mathbf{X})$, with at most one column from the right-hand side. We split this step into four further cases.
 \begin{enumerate}
     \item \text{$M$ is a submatrix of $A_{m-1,n}(\mathbf{X}\smallsetminus \{X_{ij}\}_{i\in[k]})$}. Then by the induction hypothesis, $|M|$ is a $k$-sum from $\mathcal{M}_{\times_Z^kG_{m-1,n}}^t \subseteq \mathcal{M}_{\times_Z^kG_{m,n}}^t$.
     \item  \text{$M$ has a zero row or column}. This implies that its determinant is zero.
     \item \text{The last rows of $M$ are of the form $0 \dots0\ X_{i\,m+j}$ for some $i \in [k]$}. \\More than one row of this type implies $|M|=0$. If $M$ has one row of this type then $|M| = \pm X_{i\,m+j}|M'|$ where $M'$ is a submatrix of $A_{m-1,n}(\mathbf{X}\smallsetminus \{X_{ij}\}_{i\in[k]})$. By the induction hypothesis, $|M'|$ is a $(k-1)$-sum from $\mathcal{M}_{\times_Z^kG_{m-1,n}}^{t-1}$, therefore $|M|$ is a $k$-sum from $\mathcal{M}_{\times_Z^kG_{m,n}}^t$.
    \item \text{The last column of $M$ is in the right half of $A_j(\mathbf{X})$}. Then \begin{align*}&|M| = \sum_{i \in [k]}\pm X_{ij}|M'_i| &\text{and}&& M'_i \in A_{m-1,n}(\mathbf{X}\smallsetminus \{X_{ij}\}_{i\in[k]}).\end{align*}
By induction hypothesis, these $|M'_i|$ are $(k-1)$-sums in $\mathcal{M}_{\times_Z^kG_{m-1,n}}^{t-1}$ therefore $|M|$ is a $k$-sum from $\mathcal{M}_{\times_Z^kG_{m,n}}^{t}$.
 \end{enumerate}
Conversely, for \ref{prop:ii}, suppose $f \in \mathcal{M}_{\times_Z^kG_{m,n}}^{t}$. Since $2m > t$, there exists a $j \in [m]$ such that $X_{ij}^2 \nmid f$ for all $i \in [k]$. We split this step into four further cases.
\begin{enumerate}
    \item \text{$X_{ij} \nmid f$ for all $i \in [k]$, and $X_{im+j} \nmid f$ for all $j \in [m],i \in [k]$}, i.e.~$t = \lambda$. Then $f \in \mathcal{M}_{\times_Z^kG_{m-1,n}}^{t}$ and by the induction hypothesis it is a $t$-minor of $A_{m-1,n}(\mathbf X)$, which is a submatrix of $A_{m,n}(\mathbf X)$.
    \item \text{$X_{ij} \nmid f$ for all $i \in [k]$, and $X_{i\,m+j'} \mid f$ for some $j' \in [n]$ and $i \in[k]$}. \allowbreak
    By the definition of our monomial set, there are at most $\omega \in [m]$ factors of the form $X_{i\,m+j'}$ in $f$. Therefore there is at most one $j' \in [n]$ such that $X_{i\,m+j'}$ appears $m$ times in $f$ (varying over all choices of $i$). Note that $m\geq n$ and $t\neq \lambda$ (otherwise we are in the above case), so by considering $\mathcal{M}_{\times_Z^kG_{m-1,n}}^{t}$ we do not limit our choice of factors $X_{ij}$ in $f$.
    
    If there is no such $j'$ then $f \in \mathcal{M}_{\times_Z^kG_{m-1,n}}^{t}$ (after relabelling variables) and by the induction hypothesis is a $t$-minor of $A_{m-1,n}(\mathbf X)$ which is a submatrix of $ A_{m,n}(\mathbf X)$. If such a $j'$ exists then we claim that 
    $$\frac{f}{X_{i\,m+j'}} \in \mathcal{M}_{\times_Z^kG_{m-1,n}}^{t- 1}.$$
    This follows from the uniqueness of $j'$ and our note above. Visually, in the top half of the matrix, $A_{m-1,n}(\mathbf X)$ differs from $A_{m,n}(\mathbf X)$ by one block of the form
$$\begin{array}{cc}
 X_{1\,m+1} \dots X_{1\,m+n} \\
\vdots \hphantom{XXXXX} \vdots \\
X_{k\,m+1} \dots X_{k\,m+n}\\\end{array}$$
and by our conditions, $X_{i\,m+j'}$ is the only factor appearing $m$ times (i.e.~in all blocks). Thus there exists a submatrix of $A_{m-1,n}$, $M'$ such that $f/X_{i\,m+j'}$ is a $(t-1)$-minor. Consider the matrix
$$\left(\begin{array}{c|c}
M'& \ast\\
\hline
0&X_{i\,m+j'}\\
\end{array}\right).$$
This has determinant $f$ and is clearly a submatrix of $A_{m,n}$. Therefore $f$ is a minor.
    \item $X_{ij} \mid f$ for some $i \in [k]$
Define $\alpha$ as $\{X_{ij}\}_{i \in [k]} \cap f$. Note the $i$'s in this set are distinct by the condition on $j$, and $\omega \in [m-|\alpha|]$, so considering $\mathcal{M}_{\times_Z^kG_{m-1,n}}^{t}$ does not limit our choice of factors $X_{i\,m+j}$. Then 
$$\frac{f}{\alpha} \in \mathcal{M}_{\times_Z^kG_{m-1,n}}^{t- |\alpha|}.$$ Hence there exists a submatrix $M'$ of $A_{m-1,n}$ such that $f/\alpha$ is its determinant. Consider the following matrix, with the lower right diagonal as elements in $\alpha$, 
$$\left(\begin{array}{c|ccc}
M'&& 0\\
\hline
&-X_{i_1\,j}\\
\ast&&\ddots\\
&&&-X_{i_l\,j}
\end{array}\right).$$

This has determinant $\pm f$, and since $j$ is constant, the bottom right entries can be selected from $A_{m,n}\smallsetminus M'$ in this way. Therefore this matrix is a submatrix of $A_{m,n}$ and $f$ is a minor.
\end{enumerate}
\textbf{Case 2} : $1\neq n>m$\\
There exist two permutation matrices $P, Q$ such that $P\cdot A_{m,n}(\mathbf{X})\cdot Q$ equals
\small
$$\begin{pmatrix} 
X_{1\,2m+1} \dots X_{1\,m+n} & X_{1\,1} \dots X_{1\,m} \\
\vdots \hphantom{XXXXX} \vdots&\vdots \hphantom{XXXXX} \vdots \\
X_{k\,2m+1} \dots X_{k\,m+n}&X_{k\,1} \dots X_{k\,m}\\
&& \ddots & \\
&&&X_{1\,2m+1} \dots X_{1\,m+n} & X_{1\,1} \dots X_{1\,m} \\
&&&\vdots \hphantom{XXXXX} \vdots&\vdots \hphantom{XXXXX} \vdots \\
&&&X_{k\,2m+1} \dots X_{k\,m+n}&X_{k\,1} \dots X_{k\,m}\\
\multicolumn{2}{c}{-X_{1\,m+1}\cdot \mathbf{I}_n} & \cdots & \multicolumn{2}{c}{ -X_{1\,2m} \cdot \mathbf{I}_n}\\
\multicolumn{2}{c}{\vdots} &  &\multicolumn{2}{c}{\vdots}\\
\multicolumn{2}{c}{-X_{k\,m+1}\cdot \mathbf{I}_n} &\cdots& \multicolumn{2}{c}{-X_{k\,2m} \cdot \mathbf{I}_n}\\
\end{pmatrix}$$\\
\normalsize
\begin{multline*}
=A_{n,m}\bigg(\underbrace{\{-X_{i\,m+1}\}_{i\in k}, \dots, \{-X_{i\,2m}\}_{i\in k}}_{\text{lower half terms}},\\\underbrace{\{-X_{i\,2m+1}\}_{i\in k}, \dots, \{-X_{i\,m+n}\}_{i\in k},\{-X_{i\,1}\}_{i\in k},\dots, \{-X_{i\,m}\}_{i\in k}}_{\text
{upper half terms}}\bigg)
.\end{multline*}
%
Therefore this case follows from the first after relabelling the variables. Note that this also follows from the fact that $G_{m,n} \simeq G_{n,m}$.
\end{proof}

\subsection{Computing the bivariate conjugacy class zeta function}
We follow the method in \cite[Section 4.3]{z}. We need to determine the values of
\begin{equation}\label{a}
\prod_{t=1}^{m+n-1} \frac{\|\mathcal{J}_{\times_Z^kG_{m,n}}^t(\mathbf{X})\cup w\mathcal{J}_{\times_Z^kG_{m,n}}^{t-1}(\mathbf{X})\|_{\mathfrak{p}}}{\|\mathcal{J}_{\times_Z^kG_{m,n}}^{t-1}(\mathbf{X})\|_{\mathfrak{p}}} .
\end{equation}


Let $\mathbf{x} \in W_{k(m+n)}(\mathfrak{o})$ and $w\in\mathfrak{p}$. Without loss of generality, we may assume $m \leq n$. We split the induction by cases on $t$.
\\ \phantom{X} \\
\textbf{Case 1} : $t\leq \text{min}(m,n) =m$. Then
$$\begin{aligned}
    X^t_{a\,i}, X^t_{a\,m+j} \in \mathcal{M}_{\times_Z^kG_{m,n}}^t(\mathbf{X}) \subseteq \mathcal{J}_{\times_Z^kG_{m,n}}^t(\mathbf{X}),\\
    X^{t-1}_{a\,i}, X^{t-1}_{a\,m+j} \in \mathcal{M}_{\times_Z^kG_{m,n}}^{t-1}(\mathbf{X}) \subseteq \mathcal{J}_{\times_Z^kG_{m,n}}^{t-1}(\mathbf{X}),
\end{aligned}\phantom{XX}\forall i\in[m],j\in[n],a\in[k].$$ 
By definition
$\mathbf{x} \in W_{k(m+n)}(\mathfrak{o}) \implies \exists x \in \mathbf{x} : |x|_p =1$. Then,
$$ x^t \in \mathcal{J}_{\times_Z^kG_{m,n}}^t(\mathbf{x}) \implies \|\mathcal{J}_{\times_Z^kG_{m,n}}^t(\mathbf{x}) \|_{\mathfrak{p}}=1.$$
Thus $\|\mathcal{J}_{\times_Z^kG_{m,n}}^t(\mathbf{x}) \cup w\mathcal{J}_{\times_Z^kG_{m,n}}^{t-1}(\mathbf{x})\|_{\mathfrak{p}}=1$ and $\|\mathcal{J}_{\times_Z^kG_{m,n}}^{t-1}(\mathbf{x})\|_{\mathfrak{p}}=1$.
\\ \phantom{X} \\
\textbf{Case 2} : $m<t\leq n$. Define $$M = \nu_p(x_{1\,1}, \dots, x_{k\,m}),\ N=\nu_p(x_{1\,m+1},\dots,x_{k\,m+n}),$$ the minimal valuations over subsets of $\mathbf{x}$. Note that one of $M,N$ must be zero since these subsets cover $\mathbf{x} \in W_{k(m+n)}(\mathfrak{o})$. We split this step into two further cases.
\begin{enumerate}
\item $0=M\leq N$. Then
$$\begin{aligned}
    X^t_{a\,i} \in \mathcal{M}_{\times_Z^kG_{m,n}}^t(\mathbf{X}) \subseteq \mathcal{J}_{\times_Z^kG_{m,n}}^t(\mathbf{X}),\\
    X^{t-1}_{a\,i}\in \mathcal{M}_{\times_Z^kG_{m,n}}^{t-1}(\mathbf{X}) \subseteq \mathcal{J}_{\times_Z^kG_{m,n}}^{t-1}(\mathbf{X}),
\end{aligned}\phantom{XX}\forall i\in[m],a\in[k].$$ 
Thus $\|\mathcal{J}_{\times_Z^kG_{m,n}}^t(\mathbf{x}) \cup w\mathcal{J}_{\times_Z^kG_{m,n}}^{t-1}(\mathbf{x})\|_{\mathfrak{p}}=1$ and $\|\mathcal{J}_{\times_Z^kG_{m,n}}^{t-1}(\mathbf{x})\|_{\mathfrak{p}}=1$.
\item $0=N<M$. Let $a \in [k], j\in[n]$ such that $\nu_p(x_{a\,m+j})=0$ (this exists since $N=0$), and $b\in[k],i\in[m]$ such that $\nu_p(x_{b\,k})=M$. The monomial $X^{t-m}_{b\,k}X^{m}_{a\,m+j}$ has minimal valuation among elements in $\mathcal{M}_{\times_Z^kG_{m,n}}^t(\mathbf{X})$ evaluated at $\mathbf{x}$. Since $\mathcal{J}_{\times_Z^kG_{m,n}}^t(\mathbf{X})$ contains only $k$-sums of $\mathcal{M}_{\times_Z^kG_{m,n}}^t(\mathbf{X})$ we have that $X^{t-m}_{b\,k}X^{m}_{a\,m+j}$ has minimal valuation among elements in $\mathcal{J}_{\times_Z^kG_{m,n}}^t(\mathbf{X})$ evaluated at $\mathbf{x}$. Hence
\begin{align*}
&\|\mathcal{J}_{\times_Z^kG_{m,n}}^{t}(\mathbf{x})\|_{\mathfrak{p}}=q^{-(t-m)M},\\
&\|\mathcal{J}_{\times_Z^kG_{m,n}}^{t-1}(\mathbf{x})\|_{\mathfrak{p}}=q^{-(t-1-m)M}.\end{align*}
For all $w \in \mathfrak{p}$,
\begin{align*}
\|\mathcal{J}_{\times_Z^kG_{m,n}}^{t}(\mathbf{x})\cup w\mathcal{J}_{\times_Z^kG_{m,n}}^{t-1}(\mathbf{x})\|_{\mathfrak{p}}&=\text{max}(q^{-(t-m)M},\| w\|_{\mathfrak{p}}q^{-(t-1-m)M})\\
&=q^{-(t-1-m)M}\text{max}(q^{-M},\|w\|_{\mathfrak{p}})\\
&=q^{-(t-1-m)M}\|x_{1\,1},\dots,x_{k\,m},w \|_{\mathfrak{p}}.\end{align*}
Hence
$$ \frac{\|\mathcal{J}_{\times_Z^kG_{m,n}}^t(\mathbf{x})\cup w\mathcal{J}_{\times_Z^kG_{m,n}}^{t-1}(\mathbf{x})\|_{\mathfrak{p}}}{\|\mathcal{J}_{\times_Z^kG_{m,n}}^{t-1}(\mathbf{x})\|_{\mathfrak{p}}}=\|x_{1\,1},\dots,x_{k\,m},w \|_{\mathfrak{p}}.$$
\end{enumerate}
\textbf{Case 3} : $t>n $. Define $M$ and $N$ as before. We split this step by cases.
\begin{enumerate}
\item $0=M\leq N$. Let $b \in [k], i\in[m]$ such that $\nu_p(x_{b\,k})=0$ (this exists since $M=0$), and $a\in[k],j\in[n]$ such that $\nu_p(x_{a\,m+j})=N$. Since $t>n$, the monomial $X^n_{b\,k}X^{t-n}_{a\,m+j}$ has minimal valuation in $\mathcal{M}_{\times_Z^kG_{m,n}}^t(\mathbf{X})$ evaluated at $\mathbf{x}$. Since $\mathcal{J}_{\times_Z^kG_{m,n}}^t(\mathbf{X})$ contains only linear combinations of elements in $\mathcal{M}_{\times_Z^kG_{m,n}}^t(\mathbf{X})$ we have that $X^n_{b\,k}X^{t-n}_{a\,m+j}$ has minimal valuation in $\mathcal{J}_{\times_Z^kG_{m,n}}^t(\mathbf{X})$ evaluated at $\mathbf{x}$. Hence
\begin{align*}
\|\mathcal{J}_{\times_Z^kG_{m,n}}^{t}(\mathbf{x})\|_{\mathfrak{p}}&=q^{-(t-n)N},\\
\|\mathcal{J}_{\times_Z^kG_{m,n}}^{t-1}(\mathbf{x})\|_{\mathfrak{p}}&=q^{-(t-1-n)N},\\
\|\mathcal{J}_{\times_Z^kG_{m,n}}^{t}(\mathbf{x})\cup w\mathcal{J}_{\times_Z^kG_{m,n}}^{t-1}(\mathbf{x})\|_{\mathfrak{p}}&=q^{-(t-1-n)N}\|x_{1\,m+1},\dots,x_{k\,m+n},w \|_{\mathfrak{p}}.\end{align*}

$$\text{Therefore} \phantom{X} \frac{\|\mathcal{J}_{\times_Z^kG_{m,n}}^t(\mathbf{x})\cup w\mathcal{J}_{\times_Z^kG_{m,n}}^{t-1}(\mathbf{x})\|_{\mathfrak{p}}}{\|\mathcal{J}_{\times_Z^kG_{m,n}}^{t-1}(\mathbf{x})\|_{\mathfrak{p}}}=\|x_{1\,m+1},\dots,x_{k\,m+n},w \|_{\mathfrak{p}}.$$
\item $N<M$. Similarly,
$$ \frac{\|\mathcal{J}_{\times_Z^kG_{m,n}}^t(\mathbf{x})\cup w\mathcal{J}_{\times_Z^kG_{m,n}}^{t-1}(\mathbf{x})\|_{\mathfrak{p}}}{\|\mathcal{J}_{\times_Z^kG_{m,n}}^{t-1}(\mathbf{x})\|_{\mathfrak{p}}}=\|x_{1\,1},\dots,x_{k\,m},w \|_{\mathfrak{p}}.$$\end{enumerate}

We summarise in these results in Table \ref{tableu1}.
%\small
\begin{table}[!htbp]
    \centering
    \begin{tabular}{cccc}
\toprule
     & \multicolumn{3}{c}{$t$-th factor} \\  \cline{2-4}
         & $M,N=0$&     $0=M< N$ & $0=N<M$ \\ \hline
     $1\leq t\leq m$ &1&1&1\\
     $m<t\leq n$&1&1&$\|x_{1\,1},\dots,x_{k\,m},w \|_{\mathfrak{p}}$\\
     $n<t<m+n$&1&$\|x_{1\,m+1},\dots,x_{k\,m+n},w \|_{\mathfrak{p}}$&$\|x_{1\,1},\dots,x_{k\,m},w \|_{\mathfrak{p}}$\\ \midrule
\end{tabular}%\normalsize
    \caption{Values of the $t$-th factor in the product \ref{a} at $\mathbf{x}$}
\label{tableu1}\end{table}
%\FloatBarrier
\\Therefore,
\begin{multline*}
\prod_{t=1}^{m+n-1} \frac{\|\mathcal{J}_{\times_Z^kG_{m,n}}^t(\mathbf{x})\cup w\mathcal{J}_{\times_Z^kG_{m,n}}^{t-1}(\mathbf{x})\|_{\mathfrak{p}}}{\|\mathcal{J}_{\times_Z^kG_{m,n}}^{t-1}(\mathbf{x})\|_{\mathfrak{p}}}\\
= \begin{cases}
    1&0=M =N\\
    \|x_{1\,m+1},\dots,x_{k\,m+n},w \|_{\mathfrak{p}}^{m-1}&M<N\\
    \|x_{1\,1},\dots,x_{k\,m},w \|_{\mathfrak{p}}^{n-1}&N<M.
\end{cases}
\end{multline*}

In order to compute the integral we will need the following result.
\begin{prop}[{\cite[Section 2.1]{LinsII}}]\label{Lins}
    Let $r,s \in \mathbb{C}$, Then for all $ n \in \mathbb{N}$
    \begin{align*}
&\int_{w\in \mathfrak{p}^n} |w|^r_\mathfrak{p} d\mu = \frac{q^{-n(r+1)}(1-q^{-1})}{1-q^{-r-1}},\\
&\int_{(y,\mathbf{x})\in \mathfrak{p}\times \mathfrak{p}^{(n)}}|y|_\mathfrak{p}^r\|x_1,\dots,x_n,y\|_\mathfrak{p}^s d\mu=\frac{(1-q^{-1})(1-q^{-r-n-1})q^{-r-s-n-1}}{(1-q^{-r-s-n-1})(1-q^{-r-1})},
\end{align*}
given that the integrals converge absolutely.
\end{prop}

We split the domain of integration in correspondence to the columns of the above table,
\begin{equation}\label{q}
    \mathcal{Z}^{\text{cc}}_{\times_Z^kG_{m,n}}(s_1,s_2)=(1-q^{mn-s_2})^{-1}(1+\frac{1}{1-q^{-1}}(Z_1+Z_2+Z_3)),
\end{equation} 
where
\begin{align*}
&Z_1 = \int_{(w, \mathbf{x}) \in \mathfrak{p} \times W_{km} (\mathfrak{o}) \times W_{kn} (\mathfrak{o})} |w|^{\tau}_\mathfrak{p} d\mu,\\
&Z_2 = \int_{(w, \mathbf{x}) \in \mathfrak{p} \times W_{km} (\mathfrak{o}) \times \mathfrak{p}^{(kn)}} |w|^{\tau}_\mathfrak{p} \|x_{1\,m+1}, \ldots, x_{k\,m+n}, w\|^{-(m-1)(s_1+1)}_\mathfrak{p} d\mu,\\
&Z_3 = \int_{(w, \mathbf{x}) \in \mathfrak{p} \times \mathfrak{p}^{(km)} \times W_{kn} (\mathfrak{o})} |w|^{\tau}_\mathfrak{p} \|x_{1\,1}, \ldots, x_{k\,m}, w\|^{-(n-1)(s_1+1)}_\mathfrak{p} d\mu,\\
&\text{and }\ \tau = (m+n-1)s_1+s_2-mn-2-(k-1)(m+n).
\end{align*}

By Proposition \ref{Lins}, 
\small
$$\begin{aligned}
Z_1 =& (1-q^{km})(1-q^{-kn})\\&\cdot\frac{(1-q^{-1})q^{-(m+n-1)s_1 - s_2+mn+1+(k-1)(m+n)}}{1-q^{-(m+n-1)s_1 - s_2+mn+1+(k-1)(m+n)}},
\\
Z_2 =& (1-q^{km})\\&\cdot\frac{(1-q^{-1})(1-q^{-(m+n-1)s_1 - s_2+mn+1-n+(k-1)m})q^{-ns_1-s_2+mn-n+m+(k-1)m}}{(1-q^{-ns_1-s_2+mn-n+m+(k-1)m})(1-q^{-(m+n-1)s_1 - s_2+mn+1+(k-1)(m+n)})},
\\
Z_3 =& (1-q^{kn})\\&\cdot\frac{(1-q^{-1})(1-q^{-(m+n-1)s_1 - s_2+mn+1-m+(k-1)n})q^{-ms_1-s_2+mn+n-m+(k-1)n}}{(1-q^{-ns_1-s_2+mn+n-m+(k-1)n})(1-q^{-(m+n-1)s_1 - s_2+mn+1+(k-1)(m+n)})}.
\end{aligned}$$\normalsize


Applying these to \ref{q} and substituting $T_1 \coloneq q^{-s_1}$ and $T_2 \coloneq q^{-s_2}$ (see \nameref{app}), we get that
\begin{align*}
    N^{\text{cc}}_{m,n;q}(T_1,T_2)=& \ T_1q^{m+n}q^{(k-1)(m+n)} 
\\&+\ q^{2mn}(q^{kn}+q^{km}-q^{k(m+n)})\ T_1^{m+n+1}T_2^2q^{(k-1)(m+n)} 
\\&+\ q^{mn+1}(1-q^{km}-q^{kn})\ T_1^{m+n}T_2q^{(k-1)(m+n)} 
\\&+\ q^{2mn+1}(q^nT_1^{2m+n}+q^mT_1^{2n+m})\ T_2^2q^{2(k-1)(m+n)} 
\\&-\ q^{mn}(q^nT_1^{m+1}+q^mT_1^{n+1})T_2q^{(k-1)(m+n)} 
\\&-\ T_1^{2(m+n)}T_2^3q^{3mn+1}q^{2(k-1)(m+n)} 
    \\    D^{\text{cc}}_{m,n;q}(T_1,T_2)=&\ T_1 q^{m+n}q^{(k-1)(m+n)} (1 - T_1^{m+n-1} T_2 q^{mn+1}q^{(k-1)(m+n)}) \\&\cdot(1 - T_1^n T_2 q^{(m-1)(n+1)+1}q^{(k-1)(m+n)}) \\& \cdot (1 - T_1^m T_2 q^{(n-1)(m+1)+1}q^{(k-1)(m+n)}) (1 - T_2 q^{mn}).
    \end{align*}

\begin{remark}\label{endremark}
    Unlike the representation zeta function, the conjugacy class zeta function of a $k$-fold central product is not a scaling or shift of the function associated with the original group. From Definition \ref{bivfunc}, a change to $s_2$ would influence only the power of $|\mathcal{O}:I|$, which is independent of our choice of group scheme. 
    % Moreover from \cite[\S 4.3.1]{z}, we have
    % $$\mathcal Z^{\mathrm{irr}}_{G_{m,n}}(0,s) = \mathcal Z^{\mathrm{cc}}_{G_{m,n}}(0,s)$$
    % And therefore (via Corollary \ref{ZIRR})
    % $$\mathcal Z^{\mathrm{cc}}_{\times^k_ZG_{m,n}}(0,s) = \mathcal Z^{\mathrm{cc}}_{G_{m,n}}(0,s)$$
    Any change in $s_1$ would induce a mapping $T_1 \mapsto T_1^\alpha q^\beta$. Applying this and comparing powers of $q$ gives an inconsistent set of simultaneous equations. Scaling outside the function fails similarly to a shift in $s_1$.
\end{remark}
\newpage
\section{Conclusion}
Our calculation in Section \ref{zcc} provides a novel and direct formula for the bivariate conjugacy class function, making use of techniques explored in previous sections. In particular it shows that we do not have a direct relationship in the same way as the univariate representation case (cf. Section \ref{zirr}). This raises the question of whether there is a more natural relationship between the univariate conjugacy class zeta function for central products, as no such equivalence to the bivariate conjugacy class zeta function is currently known. 

Lins mentions in \cite{LinsI} that it would be interesting to see if we can extract any other information from the bivariate zeta functions. Since the publication of her paper they have been notably used in \cite{z} to prove Theorem~\ref{Zordanstheorem}, and in \cite{rossmann2021groupsgraphshypergraphsaverage} where Rossman and Voll extended them to a family of groups associated to cographs, i.e.~the class of graphs formed from the single vertex under disjoint unions and complements. This family includes all the groups arising from the $\mathbb Z$-Lie lattice family $\mathcal F_{n,\delta}$. Moreover, their paper introduced the bivariate ask zeta functions which they used to provide an alternate proof to \cite[Theorem 1.4]{LinsII}. This seems promising for further exploration.

A natural lead on from this paper would be to investigate $k$-fold central products associated with the $\mathbb Z$-Lie lattice families $\mathcal F_{n, \delta}$ and $\mathcal H_n$ to determine whether similar or perhaps stronger relationships hold. It would also be interesting to see the extension of this theory to other types of zeta functions, such as the class number zeta function $\zeta_k$ or even general class functions such as those discussed in \ref{conjdiscussion}. Zeta functions continue to be an active and evolving area in representation theory, and we may also see new techniques similar to twist-equivalence or bivariate zeta functions, which have only been introduced relatively recently.




\newpage
\renewcommand{\bibname}{References}
%\bibliographystyle{unsrtnat}
\bibliographystyle{acm}
\bibliography{bib}
\newpage
\section{Appendix}\label{app}

\begin{minted}[
frame=lines,
framesep=2mm,
baselinestretch=1.2,
fontsize=\fontsize{9}{9},
]{python}
var('T1 T2 q qA qBN qCN qBM qCM Z1 Z2 Z3')
var('T_1m T_1n r_m r_n r_mn k r_k r_nk r_mk r_mnk')  
var('m n k')

# Our polynomial (5.2.2)
pol1 = (1 - q^(m*n) * T2)^(-1) * (1 + Z1 + Z2 + Z3)

# Domains of integration, with substitutions for readability
Z1 = (1 - q^(-k*m)) * (1 - q^(-k*n)) * qA / (1 - qA)
Z2 = (1 - q^(-k*m)) * (1 - qBN) * qCN / ((1 - qA) * (1 - qCN))
Z3 = (1 - q^(-k*n)) * (1 - qBM) * qCM / ((1 - qA) * (1 - qCM))

# Substitutions
qA = T1^(m + n - 1) * T2 * q^(m*n + 1) * q^((k - 1)*(m + n))
qBN = T1^(m + n - 1) * T2 * q^((m - 1)*n + 1) * q^((k - 1)*m)
qBM = T1^(m + n - 1) * T2 * q^((n - 1)*m + 1) * q^((k - 1)*n)
qCN = T1^n * T2 * q^((m - 1)*(n + 1) + 1) * q^((k - 1)*m)
qCM = T1^m * T2 * q^((n - 1)*(m + 1) + 1) * q^((k - 1)*n)

Z1 = Z1.subs(qA=qA)
Z2 = Z2.subs(qA=qA, qBN=qBN, qCN=qCN)
Z3 = Z3.subs(qA=qA, qBM=qBM, qCM=qCM)

# Compiling integral
pol1 = pol1.subs(Z1=Z1, Z2=Z2, Z3=Z3)

# Our denominator is already in the correct form
pol2 = pol1.numerator().simplify_full() * T1
pol3 = pol1.denominator() * T1

# In order to simplify the numerator in terms of T1 and T2,
# we need powers of q to be constants
subs = {r_n: q^n, r_n^2: q^(2*n), r_m: q^m, r_m^2: q^(2*m), r_k: q^k,
    r_nk: q^(k*n), r_mk: q^(k*m), r_mnk: q^(k*n*m), r_mn: q^(n*m), 
    r_mn^2: q^(2*n*m), r_mn^3: q^(3*n*m), T_1m: T1^m, T_1n: T1^n,      
    T_1m^2: T1^(2*m), T_1n^2: T1^(2*n), r_nk^2: q^(2*k*n), r_mk^2: q^(2*k*m)}
reverse_subs = {v: k for k, v in subs_dict.items()}

# Apply dictionary
pol2 = pol2.expand().subs(reverse_subs)

# We create our ring over T1 and T2, treating powers of q as constants
F = QQ[q, r_m, r_n, r_mn, r_k, r_nk, r_mk, r_mnk]
F = F.fraction_field()
Ring = F[T1, T2, T_1m, T_1n]
pol2 = SR(Ring(pol2)) # Symbolic ring allows resubstituion of q

# Display final result
%display latex
pol2.subs(subs)
\end{minted}

\end{document}
%\bibliographystyle{alpha}
%\bibliographystyle{amsxport}