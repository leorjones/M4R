


We begin this section with a brief corollary of Proposition \ref{prop}, before introducing \textit{Igusa-type integrals}. Building on from our expression in terms of Poincaré series, they provide a new way of computing these zeta functions. Our focus will be on the bivariate case, however this method exists and is well-established for the univariate zeta functions (cf. \cite[Section 2.2.3]{zeta1}, or \cite[Section 2.2]{voll2008}). Recall Definition \ref{bivfunc} of the bivariate zeta functions,
%
\begin{align*}\label{bivfunc2}
\mathcal{Z}_{\mathbf{G}(\mathcal{O})}^{\mathrm{irr}}(s_{1},s_{2})=\sum_{(0)\neq I\trianglelefteq\mathcal{O}}\zeta_{\mathbf{G}(\mathcal{O}/I)}^{\mathrm{irr}}(s_{1})|\mathcal{O}:I|^{-s_{2}},
\\
\mathcal{Z}_{\mathbf{G}(\mathcal{O})}^{\mathrm{cc}}(s_{1},s_{2})=\sum_{(0)\neq I\trianglelefteq\mathcal{O}}\zeta_{\mathbf{G}(\mathcal{O}/I)}^{\mathrm{cc}}(s_{1})|\mathcal{O}:I|^{-s_{2}}.
\end{align*}

\subsection{Extension to bivariate case} Proposition \ref{prop} can be extended to the bivariate representation zeta function via the specialization proved by Lins in \cite{LinsI}. Recall the relation for nilpotency class 2 (where $r$ is the rank of $\mathfrak g/\mathfrak z$),
$$\zeta^{\widetilde{\mathrm{irr}}}_{\mathbf G(\mathfrak o)}(s) = (1-q^{r-s_2})\mathcal Z^\mathrm{irr}_{\mathbf{G}(\mathfrak o)}(s_1,s_2)\bigg|_{\substack{s_1 \to s - 2 \\ s_2 \to r}}.$$
%
The following result is immediate.
\begin{coroll}\label{ZIRR} Let $\mathbf G(\mathfrak o)$ be of nilpotency class 2. Then,
$$(1-q^{r-s_2})\mathcal Z^\mathrm{irr}_{\mathbf G(\mathfrak o)}(s_1,s_2)\bigg|_{\substack{s_1 \to ks - 2 \\ s_2 \to r}}=(1-q^{r-s_2})\mathcal Z^\mathrm{irr}_{\times^k_Z\mathbf G(\mathfrak o)}(s_1,s_2)\bigg|_{\substack{s_1 \to s - 2 \\ s_2 \to r}}.$$
\end{coroll}
%
\begin{remark}
    We could extend Corollary \ref{ZIRR} to the global bivariate function $\mathcal Z^\mathrm{irr}_{\mathbf G(\mathcal O)}$ via the Euler product \eqref{biveuler} however, in this context, this provides no further insight.
\end{remark}


\subsection{Igusa-type integrals}\label{4.1}
In \cite[Section 4.1]{LinsI}, Lins shows (in an analogous way to Section 3) that both bivariate zeta functions can be expressed in terms of Poincaré series (cf. \cite[Proposition 4.7]{LinsI}). Voll demonstrates in \cite[Section 2.2]{voll2008} that we can reduce this problem to computing certain $p$-adic integrals associated with polynomials describing the degeneracy loci of our commutator matrices; specifically, these polynomials correspond to the minors of our commutator matrix. Lins extends this approach to the bivariate case (\cite[Section 4.2]{LinsI}), which we will briefly recap here.

Recall $W_d(\mathfrak o)=\mathfrak o^d\setminus \mathfrak p^d$, where $\mathfrak o$ is the completion of $\mathcal O$ at $\mathfrak p$. Define $\mathcal R(\bf X)$ as a matrix of linear forms in $\mathfrak o[\bf X]$, a generalised version of the commutator matrices $\mathcal A(\bf X)$ and $\mathcal B(\bf X)$.

For a given matrix, the \textit{$k$-minors} refer to the set of determinants for any selection of $k$ rows and $k$ columns. For example, given a matrix $\big(\alpha_{ij} \big) \in \mathrm{Mat}_{n\times m}(F)$, the $3$-minors would be the set of determinants
$$
 \begin{array}{|ccc|}
\alpha_{ap}&\alpha_{aq}&\alpha_{ar}\\
\alpha_{bp}&\alpha_{bq}&\alpha_{br}\\
\alpha_{cp}&\alpha_{cq}&\alpha_{cr}\\
\end{array} \phantom{XX}
\begin{array}{cc}
    \text{distinct } a,b,c \in [n]\\
     \text{distinct } p,q,r \in [m].
\end{array}
$$

Let $u_\mathcal R$ denote the rank of $\mathcal R(\bf X)$ over its field of fractions, and $\mathcal F^k(\mathcal R(\bf X))$ denote the ideal generated by $k$-minors of $\mathcal R(\bf X)$. Let $\mu$ denote the additive Haar measure (see, for example \cite[Section 3.1]{deitmar2013automorphic}), normalised so that $\mu(\mathfrak o^{d+1})=1$.

We define \customsmall
$$\mathcal{Z}_{\mathcal{R}}(\rho,\tau) = \frac{1}{1-q^{-1}}\int_{(w,{\bf x})\in p\times W_{d,N}(\mathfrak{o})}|w|_\mathfrak{p}^\tau\prod_{k=1}^{u_{\mathcal{R}}} \frac{\|\mathcal{F}^k(\mathcal{R}({\bf x}))\cup w\mathcal{F}^{k-1}(\mathcal{R}({\bf x}))\|_\mathfrak{p}^\rho}{\|\mathcal{F}^{k-1}(\mathcal{R}({\bf x}))\|_\mathfrak{p}^\rho}\, d\mu$$
\normalsize
so that we can express the bivariate zeta functions as follows.

\begin{prop}[{\cite[Proposition 4.8]{LinsII}}]\label{igusa} If either $c=2$ or $p>c>2$, then
\begin{align*}
\mathcal{Z}_{\mathbf{G}(\mathfrak o)}^{\mathrm{irr}}(s_1,s_2) &= \frac{1}{1-q^{r-s_2}}\Bigg(1+\mathcal{Z}_B\bigg(-\frac{s_1+2}{2},\frac{s_1+2}{2}\,u_B+s_2-h-1\bigg)\Bigg), \\
\mathcal{Z}_{\mathbf{G}(\mathfrak o)}^{\mathrm{cc}}(s_1,s_2) &= \frac{1}{1-q^{r-s_2}}\bigg(1+\mathcal{Z}_A\bigg(-(s_1+1),(s_1+1)u_A+s_2-h-1\bigg)\bigg).
\end{align*}
\end{prop}

We will use this to calculate the conjugacy class bivariate zeta function for $k$-fold central products of groups with associated $\mathbb Z$-Lie lattice from a generalisation of the family $\mathcal G_n$ (see Definition~\ref{famililes}). First we need to find the $\mathcal A$-commutator matrix.
    
